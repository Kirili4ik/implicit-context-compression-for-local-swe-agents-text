\documentclass[%
thesis=student,% bachlor's or master's thesis
coverpage=false,% do not print an extra cover page
titlepage=false,% do not print an extra title page
headmarks=true, % headmarks can be switched on or off
english,% or `german`
font=libertine, % use `libertine` font; alternatives: `helvet` / `palatino` / `times`
math=newpxtx, % math font `newpxtx`; alternatives: `ams`, `pxtx`
BCOR=5mm,% binding correction - adapt accordingly
coverBCOR=11mm% binding correction for the cover - adapt accordingly
]{tumbook}

\makeatletter %redefine some labels from the TUM template
\provideName{\@tum@examiner@}{Supervisor}{Themensteller} % or `Themenstellerin`
\provideName{\@tum@supervisor@}{Advisors}{Betreuer} % or `Advisor` / `Betreuerin`
\makeatother

\usepackage{booktabs}% for more beautiful tables
\usepackage{xcolor}

\usepackage{diffcoeff}
\usepackage{amsmath}
% \newtheorem{definition}{Definition}[section]
% \newtheorem{theorem}[definition]{Satz}
\usepackage{fvextra} 

%Literatur
\usepackage[%
    backend=bibtex, %, or `biber` on more up-to-date systems
    sortcites, % sort automatically
    sorting=nty, % sort order
    safeinputenc, % solves problems with unicode-formatted author names etc.
    citestyle=alphabetic, %
    bibstyle=alphabetic, %
    hyperref=true, % provide clickable links
    maxbibnames=3, % shorten author list for more than 3 names
    maxcitenames=3, % use at most 3 names for key
    url=false, % do not print URLs
    doi=false, % do not print DOIs
    giveninits=true,
    ]%
{biblatex}
\addbibresource{Kirill-Thesis.bib}

% HYPERLINKS AND CLEVEREF
\usepackage{cleveref}% intelligent references

% automatische Anführungszeichen
\usepackage[autostyle=true]{csquotes}


\title{Implicit Context Condensation for Local Software Engineering Agents}
%%%\subtitle{A Comprehensive Study on LLM Context Management}

\author{Kirill Gelvan}

\degree{Master of Science}% or `Bachlor of Science`
\dateSubmitted{30. November 2025}% preferably use some universally recognized date format


\examiner{Prof.\@ Dr.\@ Gjergji Kasneci}% `Themensteller`
\supervisor{Felix Steinbauer\\Igor Slinko}% `Betreuer`


\begin{document}

\frontmatter
\maketitle


\section*{Abstract}
This thesis investigates implicit context condensation for large language model based software engineering agents.
Instead of relying on extended context windows or explicit methods, it compresses long observations into a fixed set of embeddings using an in-context LLM-based encoder attached to a frozen LLM decoder.
The compressor is pretrained on general text with a mix of autoencoding and language modeling objectives and then fine-tuned on question answering, code reconstruction, or agentic trajectories derived from SWE-bench style issues.
Experiments on SQuAD and RepoQA show that the condensed representations can match or even surpass the uncompressed baselines on single-shot tasks, preserving enough detail to reconstruct natural language and source code.
In the agentic setting, context condensation enables trajectories with more steps within a fixed context budget and reduces per-step inference time, demonstrating clear efficiency gains.
However, when the agent is applied to the end-to-end setting of SWE-bench Verified, these benefits do not lead to the higher resolve rate and underperform an untrained baseline.
% Negative results for training-free and low-capacity projection baselines, together with evidence of a bottleneck in parameter-efficient fine-tuning, indicate that effective context condensation for autonomous software engineering will require higher-fidelity compression and stronger optimization than explored here.

\section*{Zusammenfassung}
Diese Arbeit untersucht die implizite Kontextkomprimierung für Software-Engineering-Agenten, die auf Large Language Models (LLMs) basieren. Anstatt sich auf erweiterte Kontextfenster oder explizite Methoden zu stützen, werden lange Observationen hierbei in ein festes Set von Embeddings komprimiert. Dies geschieht mithilfe eines In-Context-Encoders auf LLM-Basis, der an einen statischen („frozen") LLM Decoder gekoppelt ist.
Der Kompressor wird zunächst auf allgemeinen Textdaten vortrainiert (Pretraining), wobei eine Mischung aus Autoencoding- und Sprachmodellierungs-Zielen zum Einsatz kommt. Anschließend erfolgt ein Fine-Tuning auf Question-Answering, Code-Rekonstruktion oder auf Trajektorien, die von Problemen im Stil des SWE-bench abgeleitet sind.
Experimente mit SQuAD und RepoQA zeigen, dass die komprimierten Repräsentationen in Single-Shot-Aufgaben mit unkomprimierten Baselines mithalten oder diese sogar übertreffen können. Dabei bleiben genügend Details erhalten, um sowohl natürliche Sprache als auch Quellcode zu rekonstruieren. Beim Einsatz als Agent ermöglicht die Kontextkomprimierung längere Handlungssequenzen innerhalb eines festen Kontext-Budgets sowie eine reduzierte Inferenzzeit pro Schritt, was deutliche Effizienzgewinne demonstriert.
In der End-to-End-Anwendung auf SWE-bench Verified führen diese Vorteile jedoch nicht zu einer höheren Lösungsrate; das System schneidet hierbei schlechter ab als eine untrainierte Baseline.

\cleardoublepage{}

\hypertarget{toc}{}
\tableofcontents


\mainmatter{}

% Add backlink to Contents in footer next to the page number (only in mainmatter)
\renewcommand*{\pagemark}{{\hyperlink{toc}{$\leftarrow$~}\thepage}}


% ========================================
% CHAPTER FILES
% ========================================
% Each chapter is now in its own file for better organization
% ========================================
% CHAPTER 1: INTRODUCTION (MOTIVATION)
% ========================================
\chapter{Introduction}


% ========================================
% SECTION 1.1: THE CONTEXT LENGTH CHALLENGE IN LARGE LANGUAGE MODELS (LLMS)
% ========================================
\section{The Context Length Challenge in Large Language Models (LLMs)}

The ability of Large Language Models (LLMs) to effectively process long sequences of input text is fundamentally constrained by their architecture. Specifically, Transformer-based LLMs face inherent limitations due to the self-attention mechanism. Much previous research has attempted to tackle this long context issue through architectural innovations, but these efforts often struggle to overcome a notable decline in performance on long contexts despite reducing computation and memory complexity.

The long context limitation presents a significant practical challenge, particularly in complex automated scenarios. This restriction is exacerbated in software engineering (SWE) agent applications, where operational trajectories frequently involve tool calls that generate unnecessarily long outputs. For instance, custom editing tools available to the agent, such as str\_replace\_editor, are explicitly designed to truncate long command outputs, which are then marked to indicate the missing context. The LLMs' limited context length prevents them from efficiently processing the accumulated history generated by these tools.

Context compression offers a novel approach to addressing this issue, motivated by the observation that a text can be represented in different lengths in an LLM while conveying the same information. For example, the same information might be represented by a context length of 2,572 characters, 512 (sub-)words, or a compact 128 memory slots, without necessarily affecting the accuracy of the model's subsequent response.

The core goal of context condensation is precisely to leverage this potential density to enable LLM agents to execute tasks involving long chains of reasoning (or Chain-of-Thought, CoT) and more steps by condensing the environment observations. Achieving this condensation improves the model's capability to handle long contexts while offering tangible advantages in improved latency and reduced GPU memory cost during inference. For instance, empirical testing shows that compression using the In-context Autoencoder (ICAE) framework can achieve over $2\times$ to $3.6\times$ inference speedup in total time, especially in compute-intensive scenarios.


% ========================================
% SECTION 1.2: REFRAMING THE GOAL: FEATURE EXTRACTION VS. COMPRESSION
% ========================================
\section{Reframing the Goal: Feature Extraction vs. Compression}

When developing a context condensation strategy, the definition of success must be carefully framed. The approach investigated utilizes the In-context Autoencoder (ICAE), which leverages the power of an LLM to compress a long context into short compact memory slots that can be directly conditioned upon by the decoder LLM.

However, the approach should be redefined not merely as "compression," which often implies a robust, high-ratio data reduction, but rather as "summarization" or "fixed length feature extraction," due to a core methodological constraint. This constraint arises from the hardcoded, non-robust nature of the intended output length (e.g., aiming for 256 tokens in general discussion).

This reframing is critical because lossless compression typically struggles to achieve ratios exceeding $10\times$. Under high compression ratios (lossy compression), the assumption that "all tokens in the context are equally important"—an assumption intrinsically aligned with lossless autoencoding training—is violated. When the information carrier capacity is limited, the compression mechanism should ideally focus only on the most important tokens, which conflicts with the uniform coverage implied by fixed-length encoding.

The fundamental mechanism supporting this goal is the relative density of different representation spaces. The latent space of embeddings is "much denser than the discrete space of tokens," which is the underlying justification for learning context condensation. Therefore, the thesis investigates how condensing environment observations (which contain irrelevant or redundant information) into continuous representations (embeddings/memory slots) affects agent performance and efficiency when addressing the context length challenge.

Wichtige Informationen finden sich in \cref{tab:wonderful-table}.

\begin{table}[hbt]
  \centering
  \begin{tabular}{rl}
    \toprule%
    \textbf{Name}& \textbf{Place of Birth}\\ \midrule
    Gauß & Braunschweig\\
    Euler & Basel\\
    Edmonds & Washington, D.\@C.\@\\
    \bottomrule
  \end{tabular}

  \caption{A most wonderful table}%
  \label{tab:wonderful-table}
\end{table}

% ========================================
% CHAPTER 2: BACKGROUND (FUNDAMENTALS)
% ========================================
\chapter{Background}

\textbf{note: here i write explanations of things.
What is needed to understand ICAE?
But without e.g.
transformer.}

% ========================================
% SECTION 2.1: TRANSFORMER ARCHITECTURE AND POSITIONAL BIAS
% ========================================
\section{Transformer Architecture and Positional Bias}

Self-attention mechanism is well-known and described in detail in many articles.
Thus, we focus on how positional signals influence which tokens are likely to interact, and why the layout of position identifiers (position IDs) matters for context compression.

Transformers are permutation-invariant over their inputs unless augmented with positional information.
In the original formulation, absolute position encodings \cite{vaswani_attention_2017} (either fixed sinusoidal or learned) are added to token embeddings so that attention has access to token order.
Subsequent works replace addition with position-dependent biases or transformations that make attention explicitly sensitive to token distances:

\begin{itemize}
  \item Relative position representations add an index-dependent bias to the attention logits \cite{shaw_relative_2018}.
  \item Rotary position embeddings (RoPE) apply a rotation to queries and keys so that their inner product becomes a function of relative displacement \cite{su_roformer_2021}.
\end{itemize}

Formally, for queries \(Q\), keys \(K\), and values \(V\), attention often takes the form
\[
\operatorname{Attn}(Q,K,V) = \operatorname{softmax}\Big( \frac{QK^\top + B}{\sqrt{d_k}} \Big) V,\
\]
where \(B\) is a position-dependent bias.
In relative schemes, \(B_{ij} = b(i-j)\).

In RoPE, each query and key vector is multiplied element-wise by a rotation matrix that depends on its absolute position, but the resulting dot product \(\langle Q_i, K_j\rangle\) depends only on the relative distance \(i-j\).
Specifically, RoPE applies a rotation to the query and key embeddings:
\[
Q_i = R_i q_i, \quad K_j = R_j k_j,
\]
where \(R_m\) is a rotation matrix parameterized by position \(m\).
The key property is that the inner product becomes
\[
\langle Q_i, K_j \rangle = \langle R_i q_i, R_j k_j \rangle = \langle q_i, R_{j-i} k_j \rangle,
\]
which depends only on the relative offset \(j-i\).
This is achieved by constructing \(R_m\) as a block-diagonal matrix of 2D rotations with frequencies that decrease geometrically across dimensions, allowing the model to capture both short- and long-range dependencies through different frequency components \cite{su_roformer_2021}.

These mechanisms create a local inductive bias: tokens that are closer in position tend to have higher prior attention affinity.
This has direct implications for compression with special tokens ("memory", or compressed tokens): where those tokens are placed in position-ID space controls which parts of the sequence they can most easily interact with.
Empirically, assigning position IDs to minimize distance between compressed tokens and the tokens they must interface with (either source content or the downstream prompt) improves effectiveness \cite{zhao_position_2025}.

\textbf{TODO: write about it more!!!
we would talk about it later in the thesis?}

% ========================================
% SECTION 2.2: PARAMETER-EFFICIENT LLM FINE-TUNING
% ========================================
\section{Parameter-Efficient LLM Fine-Tuning}

There are many techniques to efficiently fine-tune LLMs, but we focus on Low-Rank Adaptation (LoRA) \cite{hu2021lora}.
LoRA freezes the pre-trained weight matrices and introduces trainable low-rank updates, yielding substantial parameter savings while preserving the base model’s knowledge \cite{hu2021lora}.
Consider a linear projection with base weight \(W_0 \in \mathbb{R}^{d_{\text{out}}\times d_{\text{in}}}\).
LoRA parameterizes an additive update
\[
\Delta W = B A,\quad A \in \mathbb{R}^{r \times d_{\text{in}}},\; B \in \mathbb{R}^{d_{\text{out}} \times r},\; r \ll \min(d_{\text{in}}, d_{\text{out}}),
\]
so that the adapted layer computes
\[
h = (W_0 + \alpha / r \cdot B A)\, x = W_0 x + \alpha / r \cdot B (A x),
\]
with a scalar scaling \(\alpha\) controlling the update magnitude.
Only \(A\) and \(B\) are trained; \(W_0\) remains frozen.
The trainable parameter count becomes \(r(d_{\text{in}} + d_{\text{out}})\), dramatically less than \(d_{\text{in}}\, d_{\text{out}}\) for typical ranks (e.g., tens to a few hundreds).

In Transformer blocks, LoRA adapters are commonly inserted in attention projections (query and value projections, see \cite{hu2021lora}) and optionally in output or feed-forward projections, trading off capacity and efficiency.
The benefits include:
\begin{itemize}
    \item parameter efficiency and reduced activation memory
    \item modularity—multiple task-specific adapters can be swapped on top of a single base model
    \item faster fine-tuning
\end{itemize}
Practical considerations include choosing the rank \(r\), the scaling \(\alpha\), as well as target layers to balance adaptation capacity and generalization \cite{hu2021lora}.

% ========================================
% SECTION 2.3: AGENTIC SETUP AND TOOL USE
% ========================================
\section{Agentic Setup and Tool Use}

We use "agentic" to refer to autonomous decision-making loops in which an LLM plans, invokes tools, and incorporates observations to pursue goals.
This paradigm is formalized in the ReAct (Reasoning and Acting) framework \cite{yao_react_2022}, which interleaves reasoning traces with action execution.
At time \(t\), the agent conditions on a history \(H_t = [(a_1,o_1),\dots,(a_{t-1},o_{t-1})]\) and a task-specific system prompt \(s\) to select an action \(a_t\).
The environment (or tool) returns an observation \(o_t\), which is appended to the history.
This action-observation loop continues until termination.
Concretely:
\begin{enumerate}
  \item Prompt assembly: system instructions + task + concise guidelines.
  \item Model step: the LLM proposes an action (a tool to invoke and parameters if required) and an optional justification for the action (reasoning).
  \item Tool execution: the specified tool is executed non-interactively with provided arguments.
  \item Observation: tool output (e.g.
stdout/err, JSON, file diffs, retrieval snippets, etc.) is captured.
  \item History update: \((a_t, o_t)\) is logged to \(H_t\).
  \item Termination or next step: the agent either returns a final answer or continues the loop.
\end{enumerate}

A prominent benchmark for evaluating such agentic capabilities is the Berkeley Function Calling Leaderboard (BFCL) \cite{patil2025bfcl}.
BFCL provides a standardized suite of tasks to measure an LLM's proficiency in translating natural language requests into precise, executable tool calls.
The tasks range from simple, single-function invocations to complex scenarios requiring multi-step reasoning and tool chaining.
This benchmark exemplifies the practical challenges in agentic systems, where models must correctly interpret user intent and interact with external APIs or codebases \cite{patil2025bfcl}.

\textbf{Agentic toolcall examples (from \cite{patil2025bfcl}):}
\begin{itemize}
  \item Vehicle status lookup: \texttt{vehicle.getStatus(vin="WVWZZZ...")} — returns battery level, tire pressure, and last seen location.
  \item Driving route plan: \texttt{maps.route(origin="Munich", dest="Berlin", mode="driving")} — computes ETA and step-by-step directions.
  \item Hotel search: \texttt{booking.search(city="Prague", dates="2025-11-03..05", guests=2)} — lists options with price and rating.
  \item Weather check: \texttt{weather.current(city="Warsaw")} — returns temperature, precipitation, and alerts.
\end{itemize}

\textbf{Code-oriented toolcall examples}
\begin{itemize}
  \item Code/bash execution: \texttt{execute(command="pytest -q")} or \texttt{execute(command="ls -la")} — runs unit tests using pytest or lists files with details.
  \item Symbol search: \texttt{find(name="parseUser")} — finds the definition and usages of a function in the codebase.
  \item String replacement in code: \texttt{str\_replace(file\_path="app.py", search="foo", replace="bar")} — replaces all occurrences of "foo" with "bar" in app.py.
\end{itemize}

% ========================================
% SECTION 2.4: LARGE LANGUAGE MODELS FOR CODE
% ========================================
\section{Large Language Models for Code}

Large Language Models have demonstrated significant capabilities in understanding, generating, and manipulating source code.
This proficiency stems from their training on vast corpora of publicly available code, which allows them to learn the syntax, semantics, and common patterns of various programming languages.
For an LLM, code is treated as another form of structured text, and the same sequence modeling principles that apply to natural language can be adapted for software.

The core capabilities of LLMs in the context of software engineering include:
\begin{itemize}
  \item \textbf{Code Generation:} Creating code snippets, functions, or even entire programs from a natural language description (docstring).
  \item \textbf{Code Completion:} Suggesting completions for partially written lines of code, similar to IDE autocompletion but often with more context-awareness.
  \item \textbf{Code Translation:} Migrating code from one programming language to another (e.g., Python to JavaScript).
  \item \textbf{Bug Detection and Repair:} Identifying potential bugs in a piece of code and suggesting fixes.
  \item \textbf{Code Editing:} Modifying existing code based on high-level instructions \cite{geeksforgeeks_cursor_2025}.
  \item \textbf{Code Question Answering:} Answering natural language questions about a codebase.
  \item \textbf{Agentic Code Tasks:} Autonomously planning and executing complex, multi-step software engineering tasks \cite{learncursor_agent_2025}.
\end{itemize}

These capabilities are often evaluated on benchmarks such as HumanEval \cite{chen2021evaluating} and MBPP (Mostly Basic Python Programming) \cite{austin2021program}, which test a model's ability to generate functionally correct code from specifications.

The application of LLMs to coding tasks has led to the development of powerful developer tools, such as GitHub Copilot, which are powered by models like OpenAI's Codex.
These tools act as AI pair programmers, assisting developers and increasing their productivity.
In the context of agentic systems, the ability to generate and understand code is fundamental for building autonomous agents that can interact with software environments, execute commands, and solve complex software engineering tasks.
% ========================================
% CHAPTER 3: RELATED WORK (WHAT WAS PREVIOUSLY DONE)
% ========================================
\chapter{Related Work}

\textbf{note: here i write "what was previously done". see 3.4 for that}


% ========================================
% SECTION 3.1: ARCHITECTURAL APPROACHES FOR LONG CONTEXT MODELING
% ========================================
\section{Architectural Approaches for Long Context Modeling}

Long-context modeling must address the quadratic cost of vanilla self-attention \cite{vaswani_attention_2017} while preserving task-relevant dependencies over thousands of tokens.

A useful taxonomy distinguishes:
\begin{itemize}
    \item attention variants that alter the attention operator itself;
    \item explicit compression methods that reduce the input via retrieval or summarization;
    \item implicit compression methods that learn compact, decoder-friendly representations without exposing full inputs during inference.
\end{itemize}

We briefly review each group and summarize advantages and limitations.

\paragraph{Attention variants.}
Sparse and local/windowed patterns reduce pairwise interactions to achieve sub-quadratic cost.
Windowed attention restricts each token to a fixed neighborhood and optionally augments a small set of global tokens to propagate long-range information.
Longformer combines sliding windows with learnable global tokens for long documents \cite{beltagy_longformer_2020}.
Block- and mixed-sparsity patterns (e.g., banded + random) as in BigBird provide theoretical expressivity and empirical gains on long sequences \cite{zaheer_bigbird_2020}.
Sparse Transformers \cite{child_sparse_2019} introduced fixed sparse patterns for scalable generation.
Advantages include improved memory/computation and strong local modeling.
Disadvantages include potential failures to route cross-window interactions when global tokens or connectivity patterns are insufficient, and hardware inefficiencies for irregular sparsity.

Linear-time approximations further change the attention operator.
Kernelized attention (Transformers-as-RNNs) linearizes softmax attention for autoregressive decoding \cite{katharopoulos_transformers_2020}.
Linformer projects keys/values along sequence length to attain linear complexity \cite{wang_linformer_2020}.
These methods offer asymptotic gains and longer feasible contexts.
However, they can underperform full attention on tasks requiring precise long-range interactions or exact softmax geometry.
They also introduce approximation/projection hyperparameters that affect quality.

\paragraph{Explicit compression.}
Retrieval-augmented generation (RAG) or summarization-based pipelines reduce the effective input by selecting or rewriting content before decoding.
RAG retrieves top-$k$ passages from an external corpus and conditions generation on them.
This approach improves knowledge-intensive tasks while decoupling parametric and non-parametric knowledge \cite{lewis_rag_2020}.
Abstractive summarization pre-compresses long inputs into concise proxies (e.g., PEGASUS pretraining with gap-sentence objectives) \cite{zhang_pegasus_2020}.
Benefits include controllable compute and access to external knowledge.
Drawbacks include selection bias, retrieval latency, brittleness to retrieval errors, and potential loss of details critical for downstream reasoning.

\paragraph{Implicit compression.}
Here, the idea is to produce task-adapted representations (often embeddings) that a model uses during inference, rather than the input itself.
Examples include learned soft/prefix prompts for steering frozen decoders \cite{li_prefix_2021,lester_prompt_2021}.
Other examples include tokenized memories trained to preserve answer-relevant information.
Implicit methods maintain a tight interface to the model.
They can reduce latency and memory without external retrieval.
A key challenge of implicit compression is that, by condensing input into a compact intermediate representation, some information may inevitably be lost and cannot be recovered with perfect fidelity. 
This may limit the utility of compressed representations, especially when critical details are omitted during the compression process.

Taken together, attention variants trade exactness for structure or approximation.
Explicit compression trades completeness for selection.
\textbf{TODO: this is bullshit but make a change to section 4. we like implicit -> so we found ICAE---} --- Implicit compression trades human readability for decoder-optimized compact interfaces.

% Insbesondere weisen wir auf den wunderbaren Artikel von \textcite{Edmonds:1965} und auf~\cite{GareyJohnson:1979} für weitere Hintergründe.


% ========================================
% SECTION 3.2: SOFT PROMPTING, CONTEXT DISTILLATION, AND CONTINUOUS-THOUGHT REPRESENTATIONS
% ========================================
\section{Soft Prompting, Context Distillation, and Continuous-Thought Representations}

Soft prompting is a method for conditioning large language models by learning continuous prompt vectors or prefix tokens rather than discrete text. 
These learned vectors are typically prepended to the model's input sequence and serve as a compact, trainable interface for adapting frozen decoders.
Classic methods in this category include prefix-tuning \cite{li_prefix_2021} and prompt tuning \cite{lester_prompt_2021}.
Both approaches involve optimizing a small set of continuous embeddings that steer the model toward desired behavior without full-scale model finetuning.
This parameter-efficient interface enables flexible adaptation and can be tuned for domain- or task-specific goals.
Because the prompt vectors are not constrained to map onto human-readable tokens, they can condense much more information than would be possible with standard textual prompts.

In practical, agentic settings the challenge of long or growing histories becomes acute (???).
%Models can distill the relevant information into a compact embedding or set of memory tokens.
%This reduces latency and memory overhead, as only condensed facts are passed forward during multi-step interactions.
%Examples of this pattern include ReAct-style prompting, which accumulates observations and actions over iterative reasoning steps \cite{yao_react_2022}, and Toolformer, which learns to use tools with self-supervised trajectories and requires summarizing complex action histories \cite{schick_toolformer_2023}.
%A scientific motivation for context distillation and soft prompting is that the resulting compression can be optimized directly for downstream utility, such as task success, rather than mere faithfulness to the original text.
%By learning what to keep and what to discard, these methods can mitigate biases and error propagation that affect explicit retrieval or summarization.

\textbf{CoConut} (Chain of Continuous Thought) \cite{coconut_placeholder,arxiv_2412_06769} generalizes the concept of soft prompting by moving away from natural language tokens altogether and enabling reasoning directly in latent, continuous spaces.
Instead of relying on explicit tokenization and sequence rewriting, CoConut leverages the model's own hidden states as a "continuous thought" vector.
After processing an input, the final hidden state (or a structured set of latent embeddings) is fed back as a contextual scaffold for further reasoning steps.
% Unlike prompt tuning, which fixes a learned vector prepended to all inputs, CoConut operates on an ever-evolving latent summary, tailored to the agent’s state as the interaction progresses.
Experiments show that directly reasoning in the model's own latent space improves downstream performance for tasks with extended, multi-step dependencies, outperforming classic chain-of-thought prompting.
By discarding the constraints of discrete tokenization, CoConut demonstrates that agentic LLMs can reason, plan, and retain context in a fundamentally more expressive and compact way.


% ========================================
% SECTION 3.3: THE IN-CONTEXT AUTOENCODER (ICAE) FRAMEWORK
% ========================================
\section{The In-Context Autoencoder (ICAE) Framework}

\textbf{TODO: I am very lost on where and how to write about ICAE. 
I guess here should be none, but then I need smth to compare 3 ideas from 3.1?
Then I also need ICAE in 4 and 5 somehow. Also 4.2 needs it but 4.3 is the same (expl. of AE vs LM for 50/50)}

ICAE \cite{ge_-context_2024} is closely related to the above implicit compression paradigm.
An encoder (often a LoRA-adapted copy of the base LLM) reads a long context and emits a small set of learnable \emph{memory tokens}.
A frozen decoder (the base LLM) then conditions on these tokens plus the downstream prompt to generate outputs.


ICAE differs from heuristic summarization in that the representation is optimized for decoder consumption rather than human readability.
It also differs from sparse or windowed attention in that the decoder still operates with dense attention over a small set of memory tokens.
ICAE differs from explicit RAG in that no external retrieval is required at inference \cite{beltagy_longformer_2020,zaheer_bigbird_2020,lewis_rag_2020}.
Compared to purely recurrent memory, ICAE offers direct, content-dependent access via attention over a small set of tokens.
This avoids long chains through recurrent states.



\section{Some attempts to do the same thing?}

\textbf{TODO: make a proper literature review..?}

Tobias' blogpost from openhands is not implicit, but explicit. so i have no knowledge of the same solutions using implicit compression?

Should write here about the other solutions for our problem/dataset? but what are those?
% ========================================
% CHAPTER 4: METHODS (CONCEPTUAL: ARCHITECTURE AND TRAINING)
% ========================================
\chapter{Methods}
\label{cha:methods}

% NOTE: This chapter describes the conceptual framework of applying ICAE to agentic trajectories.
% It avoids experimental specifics, results, and detailed configurations, which are covered in subsequent chapters.

% ========================================
% SECTION 4.1: ICAE FOR AGENTIC CONTEXT MANAGEMENT
% ========================================
\section{ICAE for Agentic Context Management}

The In-Context Autoencoder (ICAE) \cite{ge2023context} consists of two modules: a trainable encoder (typically a LoRA-adapted LLM) and a fixed decoder (the base LLM itself).
The encoder processes a long context and generates a fixed number of learnable memory tokens.
This design turns a long, potentially unwieldy context into a compact representation that the decoder can efficiently consume, improving latency and memory footprint while preserving fidelity for downstream tasks.
The number of memory tokens controls the compression ratio, and their placement influences how the decoder accesses the stored information \cite{ge2023context}.

Figure~\ref{fig:icae} depicts the encoder–decoder split.
The encoder ingests the full context and produces memory tokens.
The frozen decoder then receives these tokens and a prompt to generate a continuation.
During pretraining, the encoder is optimized to enable the decoder to reconstruct the original text.
During fine-tuning, the objective shifts to solving a downstream task (e.g., generating a tool call) using the compressed representation.
Parameter-efficient methods like LoRA \cite{hu2022lora} are used to adapt the encoder while keeping the base model's capabilities intact.

\begin{figure}[hbt]
  \centering
  \includegraphics[width=0.8\textwidth]{graphs/icae.jpeg}
  \caption{In-Context Autoencoder (ICAE) framework architecture.}
  \label{fig:icae}
\end{figure}

Our main contribution is the adaptation and application of this framework to an agentic setting.
In this scenario, an agent interacts with an environment over multiple turns, generating a trajectory of actions and observations.
Our method uses ICAE to compress long observations, keeping the overall context manageable without losing critical information.

% ========================================
% SECTION 4.2: TRAINING METHODOLOGY FOR AGENTIC ICAE
% ========================================
\section{Training Methodology for Agentic ICAE}

The process for training ICAE in an agentic scenario is depicted in Figure~\ref{fig:icae-agent-training-overview}.
The training data consists of pre-recorded expert trajectories, which are sequences of alternating actions and observations.

At the start of the interaction, an uncompressed user prompt is provided.
Then, a trajectory unfolds as an agent takes an action (tool call), which is sent to an environment, and receives an observation in return.
This observation becomes input for the next step.
During the trajectory, compression is applied to every long observation, ensuring that the decoder model never processes lengthy raw text.
Instead, it operates on compact embedding representations.

\begin{figure}[hbt]
  \centering
  \includegraphics[width=0.8\textwidth]{graphs/mega-1.jpeg}
  \caption{Overview of applying ICAE to agentic trajectories during fine-tuning.}
  \label{fig:icae-agent-training-overview}
\end{figure}

Figure~\ref{fig:icae-agent-training-step} illustrates a single fine-tuning step.
The observation from the environment is passed to the ICAE encoder, which produces a compressed representation.
This representation, along with the prior conversation history, is fed to the frozen decoder to generate the next tool call.
The training loss is the cross-entropy between the generated tool call and the reference action from the expert trajectory.
This loss is backpropagated through both the decoder and encoder to update only the encoder's LoRA weights, while the base model weights remain frozen.

\begin{figure}[hbt]
  \centering
  \includegraphics[width=0.8\textwidth]{graphs/mega-2.jpeg}
  \caption{A single fine-tuning step for the agentic ICAE model.}
  \label{fig:icae-agent-training-step}
\end{figure}

\subsection{Pretraining (PT)}
The first stage follows the original ICAE formulation~\cite{ge2023context}, pretraining the encoder on a large, general-purpose text corpus.
Two self-supervised objectives are used in a 50/50 mix:
\begin{enumerate}[label=(\roman*)]
    \item \textbf{Autoencoding (AE)}, where ICAE restores the original input text from its memory slots. This task is signaled by a special \texttt{<AE>} token, as conceptually illustrated in Figure~\ref{fig:icae}.
    \item \textbf{Language Modeling (LM)}, which predicts the continuation of a context to improve generalization, without the use of any special tokens.
\end{enumerate}
During this stage, only the encoder's LoRA weights are trained.
The goal is to teach the encoder to produce embeddings from which the frozen decoder can effectively reconstruct or continue text.

\subsection{Fine-Tuning (FT)}
After pretraining, the ICAE encoder is fine-tuned on a dataset of agentic trajectories.
Again, only the encoder's LoRA weights are updated.
The objective is to maximize the probability of generating the correct agent action (i.e., tool call) conditioned on the memory slots (for compressed observations) and the rest of the history.

During training, we optimize over single-step transitions.
At a timestep \(k\), the encoder compresses the observation \(o_{k-1}\).
The decoder then generates action \(a_k\) from the compressed history.
The loss from \(a_k\) is backpropagated to update the encoder's LoRA weights.
Crucially, whenever an observation exceeds a predefined threshold (e.g., 256 tokens), the encoder compresses it into a fixed set of memory tokens.
This ensures the model never processes the full raw text of long observations, allowing it to handle arbitrarily long trajectories without exceeding context limits.

For example, consider the following trajectory:
\begin{enumerate}
  \item \textbf{System prompt} (text): initial instructions and tool descriptions.
  \item \textbf{Task description} (text): user-provided issue or goal.
  \item \textbf{Action 1} (text): e.g., \texttt{bash: ls -la}.
  \item \textbf{Observation 1} (short text, $<256$ tokens): directory listing, kept as-is.
  \item \textbf{Action 2} (text): e.g., \texttt{str\_replace\_editor: view file.py}.
  \colorbox{yellow}{\textbf{Observation 2} (long text): entire file content, compressed into memory tokens.}
  \item \textbf{Action 3} (text): e.g., \texttt{str\_replace\_editor: str\_replace ...}.
  \colorbox{yellow}{\textbf{Observation 3} (long text): edit confirmation with context, again compressed into memory tokens.}
  \item \textbf{Action 4} (text): e.g., \texttt{bash: pytest}.
  \item \textbf{Observation 4} (short text): test results summary, kept as text.
  \item \textbf{Action 5} (text): \texttt{submit}.
  \item \textbf{End}.
\end{enumerate}
In this example, the encoder is applied twice (to compress observations 2 and 3, highlighted in yellow), while the decoder generates five actions.
The model then is able to predict actions from a history where long observations have been replaced by their compact memory representations.

\subsection{Training Process and Model Variants}

Figure~\ref{fig:training-process-overview} provides a comprehensive overview of the full training and evaluation pipeline, illustrating how each of our model variants is derived. The starting point for all variants is a standard, pretrained \texttt{Qwen3-8B} model~\cite{yang2025qwen3}, which we refer to as the \texttt{Baseline}. This is a ready-to-use model, not one with random weights.

The figure illustrates two parallel training paths. The first path is for our ICAE model. The \texttt{Baseline} model is used to initialize the ICAE encoder and decoder. In the Pretraining (PT) stage, we train the LoRA weights of the encoder on the SlimPama-6B dataset~\cite{weber2024redpajama}, resulting in the \texttt{ICAE-PT} model. This model is then fine-tuned on the agentic trajectories dataset, yielding the final \texttt{ICAE-PT+FT} model. Crucially, for both ICAE training stages, only the encoder's LoRA weights are updated, while the base Qwen3 model used as the decoder remains frozen.

The second path is for a comparative baseline. The original \texttt{Baseline} Qwen3 model is directly fine-tuned with LoRA on the agentic trajectories dataset. This produces the \texttt{Baseline+FT} model. This allows us to compare our two-stage ICAE approach against a standard parameter-efficient fine-tuning of a base language model on the target task.

\begin{figure}[hbt]
  \centering
  \includegraphics[width=0.8\textwidth]{graphs/overall-names.jpg}
  \caption{Full training process overview, illustrating the derivation of \texttt{Baseline}, \texttt{ICAE-PT}, \texttt{Baseline+FT}, and \texttt{ICAE-PT+FT} models.}
  \label{fig:training-process-overview}
\end{figure}
\chapter{Evaluation}
\label{cha:evaluation}

\section{Datasets}
\label{sec:datasets}

\subsection{SWE-bench}
We have chosen to use the SWE-bench \cite{jimenez2024swebench} dataset for our experiments.
It is a well known dataset for evaluating the performance of SWE agents.
It contains a large number of SWE tasks, each with a set of instructions and a set of expected outputs.
They were collected from real life GitHub issues. 
We have chosen to work with a subset -- SWE-bench Verified \cite{swebench-verified}.
It is a subset of the SWE-bench dataset that contains only the tasks that have been verified to be correct.

\subsection{Agentic Trajectories as Data}
We treat sequential action--observation interactions (trajectories) as training data.
These trajectories were obtained using a strong teacher model (e.g., Claude Sonnet 3.7) on the SWE-bench Verified dataset to produce high-quality inputs suitable for futher training.
The setup for generating these trajectories follows the SWE-smith setup closely\cite{swe-smith}.

\subsection{SQuAD}
The Stanford Question Answering Dataset (SQuAD) is a widely used reading comprehension dataset, consisting of questions posed by crowdworkers on a set of Wikipedia articles, where the answer to every question is a segment of text, or span, from the corresponding reading passage.
wriote here shortly about https://arxiv.org/pdf/1806.03822

\subsection{RepoQA \cite{liu2024repoqa}}
RepoQA \cite{liu2024repoqa} is a benchmark designed to evaluate a model's ability to understand and reason about code at the repository level.
Unlike traditional code-related tasks that focus on standalone snippets, RepoQA requires a holistic understanding of entire codebases to answer questions.
The dataset consists of question-answer pairs grounded in real-world open-source repositories, covering a wide range of topics from API usage to intricate implementation details.
This makes it an ideal benchmark for testing the limits of our ICAE model, as it assesses the ability to compress and retrieve high-fidelity information from long, structured, and semantically dense code contexts.

\section{Quality Metrics}
\label{sec:quality_metrics}

\subsection{Some Quality Metrics}
We report several metrics to evaluate our approach.
As a simple proxy metric, we measure token-wise accuracy —- averaged fraction of the guessed tokens that match the reference trajectory (note: this is measured with teacher forcing).
However, the most important metric is the number of successfully resolved issues on SWE-bench Verified, which directly reflects the model's ability to complete real-world software engineering tasks.
We also measure mean tool-call generation time to assess computational performance.

We note that measuring trajectory length is not particularly meaningful in our setting, as 8B-scale models frequently enter loops where they repeatedly call the same tool, artificially inflating trajectory length without making meaningful progress toward task completion.

write here about autoregressive accuracy and tha it is in a way better than token-wise teacher forced accuracy (in teacher forcing its about 90 and in autoregressive its about 10 so its like closer to reality)

\subsection{BLEU Score}
The Bilingual Evaluation Understudy (BLEU) score is a metric for evaluating the quality of text which has been machine-translated from one natural language to another.
BLEU's output is always a number between 0 and 1. This value indicates how similar the candidate text is to the reference texts, with values closer to 1 representing more similar texts.
It was one of the first metrics to claim a high correlation with human judgements of quality, and remains one of the most popular automated and inexpensive metrics.

In our context, we use the BLEU score to evaluate the quality of the reconstructed text after compression and decompression by the ICAE model.
A high BLEU score indicates that the reconstructed text is very similar to the original text, which means that the compression is nearly lossless.
This is particularly important for code, where even small changes can alter the semantics of the program.

% ========================================
% CHAPTER 6: EXPERIMENTS AND EVALUATION (ACTUAL SETUPS AND RESULTS)
% ========================================
\chapter{Experiments}

\textbf{note: I guess here I write not only experiments, but also the results of the experiments?}

\textbf{TODO: where to write about enc-lora+dec-lora experiments?}


%
\section{Initial Prototype Experiments: The Necessity of Training}

In the hard embedding setting, discrete tokens are represented as one-hot vectors that index the input embedding matrix.
For condensation, we compute the elementwise mean of the one-hot vectors, yielding a convex combination over the vocabulary.

In the soft-embedding setting, we remove the argmax and delete the input embedding layer so the model consumes continuous mixtures rather than token lookups.
With Qwen2.5-4B (tied input/output embedding matrices), we feed this vector directly as the next-step input (i.e., into the stack where the removed embedding layer would have produced a token embedding).
Figures~\ref{fig:ser1}–\ref{fig:ser2} explain this online injection point and its relation to the standard token pathway.

\paragraph{Online soft-embedding pathway}
We implemented the soft pathway by bypassing token sampling and the embedding lookup:
\begin{enumerate}
    \item run a normal decode step to obtain logits
    \item compute probabilities and the corresponding expected embedding
    \item insert that continuous vector directly as the next-step input
\end{enumerate}
This is done via KV-cache manipulation to inject the continuous embeddings directly into the generation pipeline.
The intent was to test whether context can be compressed into a small number of continuous vectors without any additional training or adapters (see Fig.~\ref{fig:ser1}–\ref{fig:ser2}).

\paragraph{Regenerate-LLM offline pathway}
We have thought that in the described case, generating different answers for the first embedding iteratively might be collapsing and the next results, which could decreased score in metrics.
So we tried a method that we called regenerate-llm.
Given an input sequence, we prompt the model to reproduce that sequence under teacher forcing and, at each step, record all the output embeddings.
At inference time, instead of recomputing embeddings online, we reuse the saved embeddings as the context representation.

\paragraph{Results and details}
Table~\ref{tab:avg_variants} reports SQuAD performance under these condensation strategies.
These results, together with the implementation schematics in Fig.~\ref{fig:ser1}–\ref{fig:ser2}, establish that replacing hard tokens with untrained condensed mixtures (whether via online or via offline regenetation) substantially degrades QA accuracy, motivating the trained condensation methods that follow.

%The initial approach tested replacing hard tokens with soft/averaged continuous embeddings without fine-tuning.
%These prototype experiments, using methods like KV-cache hacks and direct embedding inputs in vLLM, demonstrated that scores decreased by more than 50\% on QA tasks (e.g., SQuAD context embed F1 dropped from 0.71 to 0.17 or 0.11).
%This negative result confirmed the hypothesis that training is necessary to effectively condense context into the latent space.

\begin{table}[h]
    \centering
    \begin{tabular}{lcc}
        \toprule
        \textbf{Setting (SQuAD), context embed} &
        \textbf{Exact Match} & \textbf{F1} \\
        \midrule
        Baseline — hard tokens         & \textbf{0.58} & \textbf{0.71} \\
        Hard embedded, avg ×2          & 0.09 & 0.21 \\
        Soft embedded online, avg ×2          & 0.05 & 0.11 \\
        Soft embedded \text{Regenerate-LLM} avg ×2          & 0.07 & 0.16 \\
        \bottomrule
    \end{tabular}
    \caption{Baseline against averaging techniques (Prompt–Q–C)}
    \label{tab:avg_variants}
\end{table}

\begin{figure}[hbt]
  \centering
  \includegraphics[width=0.5\textwidth]{graphs/ser1.jpeg}
  \caption{Visualization of the "without training" approach, p1}
  \label{fig:ser1}
\end{figure}

\begin{figure}[hbt]
  \centering
  \includegraphics[width=0.3\textwidth]{graphs/ser2.jpeg}
  \caption{Visualization of the "without training" approach, p2}
  \label{fig:ser2}
\end{figure}


\section{Basic experiments with training}

Having established in \S5.1 that naively averaging ("avg, ×2") adjacent embeddings sharply degrades QA quality, we next asked whether a learned projection inserted at the embedding interface could recover performance under the same $2\times$ compression ratio.
The motivation was that, if the embedding manifold is non-linear, a trained projection might learn a geometry-preserving down-map that simple averaging cannot provide.
The baseline (Step 1) and the "soft/hard mix works" observation (Step 2) are illustrated in \ref{fig:steps1-3} and frame this question empirically.

\begin{figure}[hbt]
  \centering
  \includegraphics[width=0.5\textwidth]{graphs/steps1-3.jpg}
  \caption{Visualization of the baseline approach, step 1-3; should be redrawn}
  \label{fig:steps1-3}
\end{figure}  

\paragraph{Architectural variants}
We explored minimal-capacity projections that compress two adjacent hidden vectors into one "soft token" acceptable to the frozen decoder.
The first family was a \textbf{linear projector}, $g_\theta:\mathbb{R}^{2d}\to\mathbb{R}^{d}$, applied to $[e_{2t-1};e_{2t}]$ with optional residual gating on the arithmetic mean to stabilize scale.
The second family was a shallow non-linear MLP (one--two layers with GELU), again mapping $2d\to d$.
A third variant inserted a full BERT encoder \cite{devlin2018bert} (12 layers, 768-dimensional hidden states) to process the concatenated embeddings $[e_{2t-1};e_{2t}]$ and produce a single compressed representation, which was then projected back to the decoder's dimensionality.
This encoder-based approach provided substantially higher capacity than the shallow projections, allowing the model to learn more complex compression patterns through its multi-layer self-attention mechanism.
These designs follow the "trainable averaging" schematics shown on \ref{fig:step35}.

\begin{figure}[hbt]
  \centering
  \includegraphics[width=0.5\textwidth]{graphs/step3.5.jpg}
  \caption{Visualization of the experimental setup, step 3.5; should be redrawn}
  \label{fig:step35}
\end{figure}  

\paragraph{Training protocol and results}
All experiments used the SQuAD \cite{squad} "context-embed" setting from \S5.1, keeping Qwen3-8B frozen and training only the projection parameters via token-level cross-entropy on answer continuations.
After verifying the pipeline on a single batch, we trained on SQuAD train and evaluated on validation.
Across linear, MLP, and BERT projections, models quickly overfit but did not generalize: validation loss flattened after early improvement (\ref{fig:losses_squad_1}), and EM/F1 remained well below the hard-token baseline, never closing the large gap to the no-compression control (e.g., the $\sim 50\%$--$80\%$ relative F1 drop visible for averaging).

\begin{figure}[hbt]
  \centering
  \includegraphics[width=0.5\textwidth]{graphs/losses_squad_1.jpg}
  \caption{Validation loss curves for the SQuAD generalization experiment using linear, MLP and BERT}
  \label{fig:losses_squad_1}
\end{figure}  


\paragraph{Ablation studies}
We varied:
\begin{itemize}
    \item projection type (linear vs.\ 1- or 2-layer MLP),
    \item normalization (pre/post LayerNorm, scale-preserving residual gates),
    \item regularization (weight decay, dropout), and
    \item the decision to re-project via vocabulary space versus staying in hidden space.
\end{itemize}
We also tried unfreezing the token embedding table while keeping the transformer blocks frozen.
None of these changes altered the qualitative outcome: projections still overfit quickly and failed to surpass the baseline (\ref{fig:steps1-3}) in EM/F1.
These negative findings echo the summary on the checkpoint deck ("all our fine-tuning techniques do not recover quality").

\paragraph{Hypotheses for the failure}
Two factors appear decisive.

Firstly, we hypothesize that if the embedding manifold is not smooth, merging two embeddings into one may lose critical geometric structure that the frozen decoder relies on, making it impossible for a simple projection to preserve the information needed for downstream tasks.


Secondly, we hypothesize that the fundamental limitation is the expressive power of the overall model architecture.
Even though the BERT encoder contains 0.1B parameters (more than the projections), it lacks the capacity to learn effective compression when paired with a frozen 8B decoder.

This observation motivated our transition to the ICAE framework, where training LoRA adapters ($\approx$2\% of the 8B model's weights) provides substantially greater expressive power by modulating the decoder's internal representations, despite involving fewer trainable parameters than the full BERT encoder.


\paragraph{Summary}
The experiments illustrated in \ref{fig:step35} demonstrate that trainable projections are insufficient to recover QA performance under $\approx2\times$ compression.
These results motivated us to explore larger-scale training approaches and alternative solutions such as the ICAE framework.


%Pretrained ICAE \cite{ge_-context_2024} demonstrated the ability to decompress general texts almost perfectly.
%High BLEU scores were achieved on datasets like PWC (99.1 for Mistral-7B, 99.5 for Llama-2-7B) and SQuAD (98.1 for Qwen3-8B), indicating that memory slots retained almost all context information for contexts up to 400 tokens.
%Analysis of reconstruction errors showed patterns similar to human memorization mistakes (e.g., restoring "large pretrained language model" as "large pretrained model"), suggesting the model selectively emphasizes or neglects information based on its understanding.

\section{ICAE Pretraining and Results on General Text Reconstruction}

We evaluate ICAE autoencoding (AE) pretraining using Qwen3-8B as the base model.
During AE, the encoder compresses input contexts at a fixed $\times 4$ ratio (specifically, $1024\!\to\!256$ tokens on average), and the decoder reconstructs the original text.
We report BLEU on SQuAD contexts tested on 100 samples.
The checkpoint \texttt{pretrain\_2607\_1024\_4\_1B/checkpoint-12000} is selected as the main run and used for fine-tuning.

\begin{table}[h]
    \centering
    \small
    \setlength{\tabcolsep}{6pt}
    \begin{tabular}{lll c}
        \toprule
        \textbf{Run} & \textbf{Checkpoint (\# steps)} & \textbf{Compression} & \textbf{BLEU (mean, n=100)} \\
        \midrule
        Qwen3-8B/full (no ICAE) & 18k & $\times 1$ & 0.867 \\
        \addlinespace
        ICAE PT (pretrain\_1207) & 9k & $\times 4$ & 0.942 \\
        ICAE PT (pretrain\_1207) & 12k & $\times 4$ & \textbf{0.964} \\
        ICAE PT (pretrain\_1207) & 27k & $\times 4$ & 0.902 \\
        \addlinespace
        ICAE PT (pretrain\_2607\_1024\_4\_1B) & 9k  & $\times 4$ & 0.909 \\
        ICAE PT (pretrain\_2607\_1024\_4\_1B) & 12k (main) & $\times 4$ & \underline{0.936} \\
        ICAE PT (pretrain\_2607\_1024\_4\_1B) & 18k & $\times 4$ & 0.928 \\
        \bottomrule
    \end{tabular}
    \caption{Autoencoding (AE) reconstruction BLEU on SQuAD contexts (100-sample evaluations only). The \texttt{pretrain\_2607\_1024\_4\_1B} 12k checkpoint is the main model used for FT.}
    \label{tab:ae_bleu_squad_ours}
\end{table}

\medskip
We have also experimented with compressing 16 tokens into 4 on average instead of 1024 into 256.
It which achieved a higher BLEU (\(\approx 0.982\)). 
Notably, none of the 100-sample AE scores reach \(\approx 0.99\).
This may be acceptable for general text but could be material for code, where near-lossless reconstruction is likely a prerequisite for downstream stability.

\medskip
\noindent\textit{Example} 
Below we show a short example, where we tried to reconstruct the README.md file of the SWE-agent project.
The difference is highlighted in yellow.

\noindent\textbf{Original:}

\quad\texttt{<p align="center">}

\quad\texttt{<a href="https://swe-agent.com/latest/">}

\quad\texttt{<strong>Documentation</strong></a>\&nbsp; ...}

\noindent\textbf{Reconstructed:}

\quad\texttt{<p align="center">}

\quad\texttt{<a href="https://swe-agent.com}\colorbox{yellow}{\texttt{/agent/}}\texttt{latest/">}

\quad\texttt{<strong>Documentation</strong></a>\&nbsp; ...}

\noindent Even in the very start of the text, the difference is noticeable: the hallucinated \texttt{/agent/} path segment in the URL, which could break navigation in a coding task.


In line with internal feedback, these AE findings suggest that the current pretrain/fine-tuning mixes undertrain the model on code: AE BLEU for code should approach text-level (near 1.0) to avoid even small inaccuracies (e.g., link/variable name substitutions).



% ========================================
% SECTION 5.3: EVALUATION ON QUESTION ANSWERING TASKS (OFFLINE)
% ========================================
\section{ICAE Fine-Tuning and Results on Question Answering Tasks}

While the original ICAE work \cite{ge_-context_2024} demonstrated promising results on their proprietary PWC dataset, we sought to validate these findings on a well-established benchmark to ensure generalizability and facilitate fair comparison with existing approaches.
To address the limitation of the authors'-crafter PWC dataset, we conducted fine-tuning experiments on the Stanford Question Answering Dataset (SQuAD) \cite{squad}, a widely recognized benchmark in the question answering literature that provides standardized evaluation protocols and enables reproducible comparisons.

The fine-tuning procedure follows the encoder-decoder structure established during pretraining, with modifications to accommodate the question answering objective.
During fine-tuning, we feed compressed context representations produced by the encoder into the decoder, concatenated with uncompressed question tokens.
The decoder then generates answer from this mixed representation, while throught backpropagation we update the LoRA weights of the encoder.
This setup encourages the encoder to learn compression strategies that preserve information necessary for the decoder to extract answers from.
We employ LoRA fine-tuning \cite{hu2021lora} with identical hyperparameters to those used in the original ICAE pretraining to maintain consistency and enable fair comparison.

To establish a comprehensive evaluation baseline, we compare four distinct model configurations that systematically vary the training procedure and compression strategy.
First, we evaluate the base Mistral-7B model \cite{mistral7b} without any fine-tuning to establish the zero-shot performance ceiling.
Second, we construct a LoRA fine-tuned baseline where we apply LoRA fine-tuning directly to Mistral-7B on the SQuAD dataset without any compression mechanism, thus representing the standard approach without context condensation.
This baseline operates without an encoder-decoder structure, functioning as a conventional LLM fine-tuned for question answering at full context length.
Third, we evaluate the ICAE model fine-tuned on PWC as provided by the original authors \cite{ge_-context_2024}, which represents their reported best configuration.
Finally, we train our own ICAE variant by fine-tuning the pretrained encoder-decoder architecture on SQuAD using identical training code and hyperparameters to those employed by the authors for PWC fine-tuning.
This parallel setup enables direct comparison while controlling for implementation differences.

Table~\ref{tab:icae_squad} presents the evaluation results across all four configurations, measured using Exact Match (EM) and F1 scores on the SQuAD validation set.
The compression ratio for ICAE variants averages approximately $1.7 \pm 0.7$, meaning contexts are condensed to roughly 60\% of their original length while maintaining the compressed representation.
\begin{table}[h]
    \centering
    \begin{tabular}{lccc}
        \toprule
        \textbf{Model} &
        \textbf{Compression} &
        \textbf{Exact Match} &
        \textbf{F1} \\
        \midrule
        Mistral-7B (no FT)          & ×$1$         & 49 & 68 \\
        LoRA-FT baseline            & ×$1$         & \underline{59} & \underline{65} \\
        ICAE FT (PwC, authors)       & ×$1.7\pm0.7$ & 41 & 57 \\
        ICAE FT (SQuAD, ours)              & ×$1.7\pm0.7$ & \textbf{69} & \textbf{73} \\
        \bottomrule
    \end{tabular}
    \caption{ICAE averaging on SQuAD}
    \label{tab:icae_squad}
\end{table}

Several important observations emerge from these results.
First, our ICAE fine-tuning on SQuAD achieves the highest performance across both metrics, with an F1 score of 73 and Exact Match of 69, representing substantial improvements over both the untrained baseline and the LoRA fine-tuned control.
Notably, this performance gain occurs despite operating under approximately 2× compression, suggesting that the learned compression strategy successfully preserves task-critical information while reducing computational overhead.

Most strikingly, the compressed ICAE model not only outperforms the uncompressed baseline (already valuable given the computational savings) but also surpasses the uncompressed LoRA fine-tuned baseline by 8 F1 points (73 versus 65).
This result is quite unexpected: compression typically implies information loss, yet here the compressed model demonstrates superior performance to its uncompressed counterpart trained with identical LoRA fine-tuning procedures.
We hypothesize that this advantage stems from the compression mechanism's ability to filter and retain only the most importnat information from the context, effectively introducing a beneficial inductive bias.
This might help the model focus on task-relevant features while discarding noise or redundant details.

Moreover, the ICAE model fine-tuned on PWC exhibits notably lower performance than all other configurations, achieving F1 and EM scores of 57 and 41 respectively.
This result indicates that the PWC-fine-tuned model fails to generalize effectively to the SQuAD evaluation distribution, despite achieving strong performance on its training domain.
This performance gap suggests potential overfitting to the specific characteristics of the PWC dataset, which may have properties that do not transfer well to standard question answering benchmarks.

These results provide encouraging evidence for the viability of applying ICAE-based compression to downstream tasks beyond the original evaluation domain.
The strong performance on SQuAD, combined with the observed compression benefits, motivated our subsequent investigation of the ICAE framework in more complex, agentic settings where context length presents significant computational challenges.

\section{Checking code data with RepoQA dataset \cite{liu2024repoqa}}
To further assess the capabilities of our ICAE model, particularly its ability to handle long, structured, and highly technical contexts, we evaluated it on the RepoQA dataset \cite{liu2024repoqa}.
This experiment was designed to test the hypothesis that ICAE can effectively compress entire code repositories while preserving the granular details necessary for complex reasoning and question answering about the code.
Given that software development tasks often require a deep understanding of a large codebase, this evaluation serves as a critical stress test for our context compression approach.

\section{[Very main part?] ICAE Fine-Tuning and Evaluation on SWE-bench Verified}

This section describes the main experiment of the thesis and discusses its results.
The conceptual methodology for applying ICAE to agentic trajectories is detailed in Chapter~\ref{cha:methods}.
Figures~\ref{fig:icae-agent-training-overview} and~\ref{fig:icae-agent-training-step} in that chapter illustrate the training process, where the model is fine-tuned on expert trajectories to predict tool calls from a history of compressed observations.
Here, we evaluate the performance of this method on the SWE-bench Verified dataset.

\subsection{Experimental Setup}
We reimplemented the ICAE framework from scratch, building upon the original architecture \cite{ge_-context_2024} with several modifications for improved efficiency and reproducibility.
Our implementation uses the Qwen3 model family (specificaly Qwen3-8B) as the base LLM, with LoRA adaptation applied to the attention matrices (q\_proj and v\_proj) using a rank of 128.
To make things simpler, we explicitly disable long-form "thinking" during decoding by inserting the token sequence \texttt{\textbackslash think \textbackslash think} with a newline marker (\texttt{\textbackslash n}) in between.
This is exactly how the authors of \cite{qwen3} recommend disabling thinking.
This prefix makes the model "believe" it has already produced intermediate thoughts (empty in this case), while in fact no additional content is generated.
It should also be noted that the authors of \cite{ge_context_2024} have only worked with older models (such as Llama2 and Mistral-v0.1), and only publish the weights of the latter.

\paragraph{Pretraining.}
This pretraining was performed on the dataset SlimPajama-6B \cite{pajama6b}\footnote{\url{https://huggingface.co/datasets/DKYoon/SlimPajama-6B}} using a combination of autoencoding and language modeling objectives.
It is a common text dataset, that is used for pretraining LLMs, consisting of 6 billion tokens (a random 1\% of the original 627B tokens).
The dataset that authors used in their original paper "The Pile" was unavailable.

%\begin{figure}[hbt]
%  \centering
%  \includegraphics[width=0.9\textwidth]{graphs/pt_losses.png}
%  \caption{Pretraining loss curves averaged by trajectory}
%  \label{fig:pt_losses}
%\end{figure}

\paragraph{Fine-tuning.}
After pretraining, ICAE is fine-tuned using trajectories from SWE-bench Verified. Fine-tuning uses a memory size of 256 tokens.
If the observation is longer than 1024 tokens, we apply the encoder multiple times, preserving the compression ratio, following the authors of \cite{ge_context_2024}.
On figure \ref{fig:ft_losses} you can notice the sudden drop of the loss.
This is phenomenon found in \cite{PI}.
It appears due to the modification of positional encodings for the memory tokens.
We see the effect being the same as described by the authors.
In our experiments, it only appers if we apply the positional encodings manipulations.

\begin{figure}[hbt]
  \centering
  \includegraphics[width=0.9\textwidth]{graphs/ft_losses.png}
  \caption{Fine-tuning loss curves}
  \label{fig:ft_losses}
\end{figure}

\paragraph{Training Infrastructure.}
Training was performed on a single NVIDIA H200 GPU, requiring approximately 1 day and 15 hours for pretraining and 3 days for fine-tuning due to the computational complexity of the autoencoding objective and resulting lack of effective batching opportunities.
Detailed hyperparameters and training configurations are provided in Appendix~\ref{app:training_details}.

\subsection{Tools and Interaction Protocol}
We follow the SWE-smith setup for SWE-bench Verified \cite{jimenez2024swebench}, where the assistant agent interacts with the environment through a minimal toolset and a fixed protocol. 
Concretely, the available tools are a shell interface (\texttt{bash}), a submission tool (\texttt{submit}), and a custom editing utility (\texttt{str\_replace\_editor}).
These tools generate the observations that accumulate within the trajectory context.
The exact SWE-smith prompt (and the tools description) is provided in Appendix~\ref{app:swe-smith-prompt}.
It should be noted that this method does not use function calling functionality of the model. 
While some models support a special format for function calling (e.g. additional fields for the tools descriptions etc.), in this case we just use the tools as described in the prompt.

\paragraph{The \texttt{str\_replace\_editor}.}
This is a stateful file editor that supports viewing, creating, and editing files with precise, line-exact operations.
Its state persists across steps, enabling consistent multi-edit workflows. 
The interface exposes the following commands: \texttt{view}, \texttt{create}, \texttt{str\_replace}, \texttt{insert}, and \texttt{undo\_edit}. 

For deterministic edits, \texttt{str\_replace} requires \texttt{old\_str} to match exactly one or more consecutive lines in the target file, including whitespace.
The \texttt{new\_str} content replaces the matched block.
The \texttt{insert} command appends \texttt{new\_str} after a specified line number.

\subsection{Compression Strategy}
The encoder is only applied if all of the next are true:
\begin{itemize}
  \item The text is an observation (i.e. response from the environment, not the action)
  \item The text is longer or equal than 256 tokens
\end{itemize}
The actions are short and do not require compression, while the observations can be long and complex.
Due to the nature of the ICAE framework, the compression produces a fixed number of memory tokens, which we fix at 256.
Applying the encoder to the shorter texts would be a waste of resources and would create a mismatch between the training and the validation settings.

\subsection{Results}
The results of our experiments are presented in Table~\ref{tab:qwen_icae_variants_absolute}. The table compares several model configurations across five oder is only applied if akey metrics:
\begin{itemize}
    \item \textbf{Encoder}: The model used to compress observations. "—" indicates no encoder and thus no compression.
    \item \textbf{Decoder}: The model used to generate the next tool call. We test the base Qwen3-8B ("Qwen"), a LoRA fine-tuned version ("Qwen-LoRA-FT"), and a fully fine-tuned version ("Qwen-Full-FT").
    \item \textbf{Accuracy}: The token-wise accuracy on the SWE-bench Verified test set.
    \item \textbf{Resolved}: The number of issues successfully resolved out of 500.
    \item \textbf{Time}: The mean time in seconds to generate a tool call.
\end{itemize}

We first establish two naive baselines to demonstrate the importance of retaining observation context.
The "del long obs-s" approach discards any observation exceeding 256 tokens, while "del all obs-s" removes all observations entirely.
As shown in Table~\ref{tab:qwen_icae_variants_absolute}, both methods result in a drastic drop in performance, with almost no issues resolved.
While they significantly reduce generation time by shortening the context, their failure highlights that observations are critical for task success, motivating the need for more sophisticated context management techniques like compression.

Next, we evaluate three uncompressed baseline models to set performance targets.
The fully fine-tuned Qwen3-8B model ("Full-FT") achieves the highest performance, resolving 86 issues and setting the upper bound for this architecture.
The LoRA fine-tuned variant ("LoRA-FT") provides a more parameter-efficient alternative, resolving a respectable number of issues.
The base Qwen3-8B model without any fine-tuning ("Qwen") serves as the most direct point of comparison for our ICAE models, as they use this same frozen model as the decoder.
It resolves 26 issues, establishing a solid baseline for an off-the-shelf model on this task.

%\begin{table}[h]
%    \centering
%    \setlength{\tabcolsep}{6pt}
%    \begin{tabular}{llcc}
%        \toprule
%        \textbf{Encoder} & \textbf{Decoder} & \textbf{Accuracy} & \textbf{Mean tool-call time (s)} \\
%        \midrule
%        % --- Baseline encoder
%        —                        & Full-FT   & 0.9484 & 1.24 \\
%        —                        & LoRA-FT   & 0.9118 & 1.24 \\
%        —                        & Qwen           & 0.8967 & 1.23 \\
%        \addlinespace
%        % --- Ablations
%        del long obs-s             & Qwen           & 0.8873 & 0.44 \\
%        del all obs-s              & Qwen           & 0.8802 & 0.39 \\
%        \addlinespace
%        % --- ICAE (Qwen pretrained) encoder
%        ICAE (LoRA-PT w/ Full-FT)   & Full-FT   &  0.9219    &  — \\
%        ICAE (LoRA-PT w/ Qwen)    & Qwen           &  0.8808 & \textbf{1.12 (0.31+0.81)} \\
%        \addlinespace
%        % --- ICAE (Qwen-LoRA-FT) encoder
%        ICAE (LoRA-FT)         & Full-FT   &  ?   & —  \\
%        ICAE (LoRA-FT)         & LoRA-FT   & 0.9263   & —  \\
%        ICAE (LoRA-FT)         & Qwen           & 0.9020 & — \\
%        % bad-seed ICAE (LoRA-FT)         & Qwen           & 0.8918 & — \\       
%
%        
%        \bottomrule
%    \end{tabular}
%    \caption{No think bug table. Qwen and ICAE future variants. FT=FineTuning, PT=PreTraining}
%    \label{tab:icae_variants}
% \end{table}

\begin{table}[h]
  \centering
  \small
  \setlength{\tabcolsep}{4pt}
  \renewcommand{\arraystretch}{1.05}

  \begin{tabular}{|ll|ccc|}
      \hline
      \textbf{Encoder} & \textbf{Decoder} & \textbf{Acc. $\uparrow$} & \textbf{Resolved (/500) $\uparrow$} & \textbf{Time (s) $\downarrow$} \\
      \hline
      \multicolumn{5}{|l|}{\hspace{1em}\textit{Naive Baselines}} \\
      \hline
      del long obs-s            &     Qwen       & 0.8873 & 1         & 0.44                      \\
      del all obs-s             &     Qwen       & 0.8802 & 0         & 0.39                      \\
      \hline
      \multicolumn{5}{|l|}{\hspace{1em}\textit{Baselines}} \\
      \hline
      —                         & Full-FT   & \textbf{0.9484} & \textbf{86}              & 1.24                      \\
      —                         & LoRA-FT   & 0.9118 & 10 (or 25 with more layers)       & 1.24                      \\
      —                         & Qwen      & 0.8967 & \underline{26}                    & 1.23                      \\
      \hline
      \multicolumn{5}{|l|}{\hspace{1em}\textit{ICAE-FT Compression}} \\
      \hline
      ICAE (LoRA-FT)       & Qwen      & 0.9020             & 11 - is it broken or not?       & \textbf{1.12 (0.31+0.81)}    \\
      ICAE (LoRA-FT)       & LoRA-FT   & \underline{0.9263} & 3 (lora is broken? 10)          & 1.13                      \\
      \hline
      \multicolumn{5}{|l|}{\hspace{1em}\textit{ICAE-PT Compression (Ablation)}} \\
      \hline
      ICAE (LoRA-PT w/ Full-FT) & Full-FT   & 0.9219 & —         & —                         \\
      ICAE (LoRA-PT w/ )        & Qwen      & 0.8808 & —         & —                         \\
      \hline
  \end{tabular}
  \caption{No think bug table. ICAE variants. All encoders and decoders are Qwen3-8B. FT=FineTuning, PT=PreTraining}
  \label{tab:qwen_icae_variants_absolute}
\end{table}


The core of our experiment tests ICAE with an encoder fine-tuned on SWE-bench trajectories.
When pairing the ICAE-FT encoder with the base Qwen decoder, we observe a slight improvement in token-wise accuracy over the uncompressed Qwen baseline (0.9020 vs. 0.8967) and a modest 10\% reduction in generation time.
However, this configuration sees a significant drop in task performance, resolving only 11 issues compared to the baseline's 26.
A similar trend holds when using a LoRA-FT decoder, where accuracy increases but the resolved rate plummets.
This suggests that while compression is efficient, it loses critical information, and that token-wise accuracy is a poor proxy for end-to-end task success in this agentic setting.

We also experimented with an ICAE encoder that was only pretrained on general text (ICAE-PT) and not fine-tuned on SWE-bench data.
Our hypothesis was that a general-purpose compressor could be a shareable artifact, applicable to new domains without task-specific fine-tuning.
However, this ablation proved unsuccessful.
The models using the ICAE-PT encoder performed worse than even the naive baselines in terms of token accuracy and failed to resolve any issues.
This demonstrates that for a complex domain like software engineering, the encoder must be fine-tuned on in-domain data to learn what information is important to preserve during compression.

The key negative result is the sharp decline in resolved issues when using ICAE-FT compression (11) compared to the uncompressed baseline (26).
We hypothesize that this performance degradation is a direct result of information loss during the compression stage.
As demonstrated in our pretraining evaluation, the autoencoding reconstruction is not lossless.
For SWE-bench tasks, subtle details in observations—such as exact file paths, variable names, or specific error messages—are often critical for making correct decisions.
The ICAE encoder, even after fine-tuning, likely discards or corrupts some of this vital information.
While the compression may be sufficient for preserving the general semantics of the text, it fails to retain the high-fidelity details required for complex, multi-step coding tasks, ultimately leading to agent failure.


Efficiency Results: ICAE \cite{ge_-context_2024} compression led to measurable efficiency improvements, achieving a theoretically 10\% faster mean tool-call generation time than the vanilla baseline (e.g., 1.12s vs 1.23s).
It took on average 0.31s to compress the observation and 0.81s to generate the next tool call.
It should be noted that all the experiments in the table do not use KV-caching.
More on that is in Chapter 6 limitations section.


\begin{figure}[hbt]
  \centering
  \includegraphics[width=0.9\textwidth]{graphs/bleu1.png}
  \caption{BLEU-1 scores comparison}
  \label{fig:bleu1}
\end{figure}


% ========================================
% SECTION 5.x: DISABLING THINKING EXPERIMENTS
% ========================================
\section{Disabling Thinking Experiments}
\label{sec:disabling_thinking}

\textbf{TODO: maybe we push it to appendix?}

The Qwen3 model family provides a mechanism to toggle its "thinking" mode, which is designed for complex reasoning.
For our agentic experiments, where a tool-call generation is prioritized, we decided to disable this feature not to overcomplicate the pipeline.
The official recommendation for disabling thinking involves prefixing the generation prompt with a specific token sequence, \texttt{<think>\textbackslash n\textbackslash n</think>}, which signals to the model that the "thinking" step has already occurred and is empty.
Crucially, for multi-turn interactions, this prefix should be ephemeral: it is added for generation and then omitted from the conversation history to prevent it from influencing subsequent turns.

In an early stage of our experiments, we explored the model's sensitivity to this prompting convention by deviating from the recommended protocol.
Instead of removing the thinking prefix from the history, we allowed it to accumulate with each agent step.
This resulted in a setup where the context for generating action \(a_k\) contained not only the history of actions and observations but also \(k-1\) instances of the thinking-disabling prefix.
While unintentional, this created a distinct experimental condition that we analyzed for its impact on model performance.

Table~\ref{tab:thinking_variants} compares the performance of key model configurations under both the official "Clean" prompting protocol and our "Cumulative" prompting experiment.

\begin{table}[h]
    \centering
    \small
    \caption{Comparison of prompting strategies for disabling thinking in Qwen3-8B. "Cumulative" refers to accumulating the `\texttt{<think>...}` prefix in the history, while "Clean" follows the official recommendation of removing it after each step.}
    \label{tab:thinking_variants}
    \begin{tabular}{llc}
        \toprule
        \textbf{Configuration} & \textbf{Prompting Strategy} & \textbf{Accuracy $\uparrow$} \\
        \midrule
        Baseline (Qwen) & Cumulative & 0.9000 \\
        Baseline (Qwen) & Clean & 0.8967 \\
        \addlinespace
        ICAE-FT + Qwen & Cumulative & 0.9089 \\
        ICAE-FT + Qwen & Clean & 0.9020 \\
        \bottomrule
    \end{tabular}
\end{table}

The results present an unexpected finding.
The "Cumulative" prompting strategy, despite polluting the context with repetitive, non-semantic tokens, yielded slightly higher token-wise accuracy compared to the "Clean" approach for both the baseline Qwen model (0.9000 vs. 0.8967) and the ICAE-FT compressed model (0.9089 vs. 0.9020).
Note, that we, of cource, do not calculate the accuracy on the thinking tokens.
We hypothesize that the repeated, structured nature of the accumulated prefixes might enable the model's attention mechanism to process the context more efficiently, or that the model learns to largely ignore these predictable tokens, leading to a marginal improvement in next-token prediction.

Anyways, all other experiments reported in this work, including the main results in Table~\ref{tab:qwen_icae_variants_absolute}, were conducted using the official "Clean" prompting methodology to ensure the validity and reproducibility of our findings.
\chapter{Discussion}
\label{cha:discussion}

This chapter provides an in-depth analysis of the experimental results and explores potential explanations for the observed performance patterns.
We examine the factors that may contribute to the degradation in agentic task performance, despite strong results on static benchmarks.
The discussion is structured around several hypotheses that collectively shed light on the challenges of applying compression methods to multi-turn, interactive software engineering tasks.

\section{Analysis of Performance Degradation}

While our compression approach demonstrates promising results on static benchmarks such as SQuAD and RepoQA, the performance on the primary agentic task (SWE-bench Verified) reveals significant challenges.
This section explores potential reasons for this degradation.

\paragraph{Potential Reasons for Performance Degradation.}
We hypothesize several factors may contribute to the degradation in task resolution performance:
\begin{enumerate}
    \item \textbf{Loss of Agentic Capabilities}: The pretraining stage on a general text corpus may weaken the planning or tool-calling capabilities of the base model that are essential for agentic tasks.
    While fine-tuning on agentic data recovers some task-specific abilities, such as code generation (as evidenced by the strong performance on RepoQA in Section~\ref{sec:eval_repoqa}), it may not be sufficient to restore the full spectrum of agentic competence.

    \item \textbf{Imperfect Reconstruction Fidelity}: As discussed in Section~\ref{sec:icae_pretraining_results}, the autoencoding pretraining does not achieve perfect reconstruction (i.e., BLEU scores do not approach $\ge 0.99$).
    In software engineering tasks, even minor inaccuracies in compressed observations can accumulate over a multi-step trajectory and lead to critical failures (see the illustrative example in Section~\ref{ex:ae-readme-recon}).

    \item \textbf{Unsuitability of LoRA-FT for Agentic Tasks}: The use of parameter-efficient fine-tuning via LoRA may be insufficient for adapting a model to complex agentic workflows.
    Our results show that the \texttt{Baseline+LoRA-FT} model also performs poorly compared to both the fully fine-tuned and simple baselines, suggesting that LoRA may not be an effective strategy for this specific task, regardless of compression.
    Notably, we have found no research done using LoRA for agentic trajectories, only full-parameter fine-tuning (which performs the best in our experiments).

    \item \textbf{The Challenge of Multi-Turn Interaction}: In single-shot NLP tasks like SQuAD or RepoQA, the context is static and complete.
    In contrast, an agentic task is dynamic: the action at step \(k\) influences the observation at step \(k+1\).
    It is impossible to determine at the moment of compression which pieces of information from an observation will become critical at a future step \(k+i\).
    The compression process, optimized for immediate reconstruction, may discard seemingly unimportant details that are essential for long-term planning and task success.
\end{enumerate}

% ========================================
% CHAPTER 6: LIMITATIONS AND FUTURE WORK (WHAT CAN WE NOT COVER)
% ========================================
\chapter{Limitations and Future Work}
\label{cha:limitations}

\section{Limitations of Fixed-Length Context Condensation}
\label{sec:limitations}

A core methodological limitation of the investigated approach is its reliance on a fixed number of memory tokens (e.g., 256) for context condensation. This design choice imposes a hardcoded, fixed compression ratio. For instance, a model configured with 256 memory tokens and a 4x compression ratio can only process a maximum of 1024 tokens of context at once. For longer inputs, such as an observation of 10,000 tokens, the condensation process would need to be applied iteratively in a loop. This would likely be slow and undermine the efficiency gains of the approach.

This fixed-length strategy is based on the assumption that all tokens in the context are equally important, which aligns with lossless autoencoding. However, this assumption becomes problematic at the high, lossy compression ratios required for significant context reduction. Experimental results confirm that performance attenuates or fails at high compression ratios (e.g., beyond 15x or 31x).


\section{Limitations of KV-caching}
\label{sec:kv_caching}
A notable limitation of our experimental setup is the exclusion of Key-Value (KV) caching, a standard optimization for autoregressive inference in Transformer models.
For methodological simplicity and to isolate the effects of context compression, in our experiments we recomputed the full attention state at each decoding step.

However, the ICAE framework is fully compatible with KV-caching.
The continuous embeddings produced by the encoder can be treated as a fixed prefix, and their corresponding key-value states can be pre-computed and cached.
Subsequent token generation would then reuse these cached states, significantly improving the absolute speed of the decoding process.
While KV-caching is essential for production deployment to achieve practical inference speeds, it was not necessary for our comparative evaluation.

\section{Constraints on Computational Resources}
\label{sec:comp_resources}

\subsection{Model Scale}
Due to computational and time constraints, our experiments were confined to the Qwen3-8B model. While evaluating on the full 500-issue SWE-bench Verified dataset provides sufficient statistical power for robust comparisons at this scale, an important direction for future research is to investigate ICAE's effectiveness on larger models like Qwen3-32B. It remains an open question whether larger models, with their increased capacity, can better leverage compressed representations or if they are more sensitive to information loss during compression.

\subsection{Full Fine-Tuning and the LoRA Bottleneck}
Our experiments with ICAE exclusively utilized LoRA for parameter-efficient fine-tuning, consistent with the original work.
However, our \texttt{baseline} experiments revealed a critical limitation: the LoRA-tuned baseline significantly underperformed the frozen base model in the agentic setting, whereas the fully fine-tuned model established the high-performance upper bound.
This "LoRA bottleneck" suggests that low-rank adaptation may be insufficient for learning the complex, multi-step reasoning required for autonomous software engineering, regardless of context compression.
Consequently, applying full fine-tuning to the ICAE encoder is not merely an optimization but likely a necessity for this domain.
Our open-sourced codebase facilitates this, and we identify it as the most important next step for future research.

\subsection{Disabled Reasoning}
For methodological simplicity, we disabled the Qwen3 decoder model's reasoning(i.e. "thinking") mechanism, which is designed to improve performance on complex reasoning tasks. Enabling this feature could prove highly beneficial, as it might allow the model to iteratively reason over the compressed knowledge stored in memory slots, potentially leading to better decision-making. Given that chain-of-thought reasoning has been shown to dramatically improve performance on various benchmarks, exploring the interaction between compressed context and explicit reasoning steps is a critical avenue for future research.


\section{Open Source Contributions and Reproducibility}
\label{sec:open_source}

We reimplemented the ICAE framework from scratch, as the original authors' code was outdated and difficult to adapt to our experimental needs.
Our implementation is modular and provides separate training pipelines for both pretraining (PT) and fine-tuning (FT).
We support pretraining on general text datasets and fine-tuning on question-answering, repo-qa tasks and agentic trajectories.

To advance the field of context compression for software engineering agents, we release our complete implementation, including pretrained models achieving 95\% reconstruction BLEU.
To ensure full reproducibility, we provide our complete implementation\footnote{\url{https://github.com/JetBrains-Research/icae}}, full Weights \& Biases experiment logs for pretraining\footnote{\url{https://wandb.ai/kirili4ik/icae-pretraining}} and fine-tuning\footnote{\url{https://wandb.ai/kirili4ik/icae-swebench-finetune}}, and model checkpoints\footnote{\url{https://huggingface.co/Kirili4ik/icae}}.
Our comprehensive release includes all training configurations, hyperparameters, and experiment logs, enabling future researchers to reproduce our results and build upon this work.
This open-source release provides the tools and transparency necessary for scientific progress in context management for LLMs.
\chapter{Conclusion}
\label{cha:conclusion}

This thesis examined implicit context condensation for software engineering agents, instantiating the approach with an In-Context Autoencoder (ICAE) built on Qwen3-8B and evaluating it on question answering (SQuAD), code reconstruction (RepoQA), and end-to-end agentic SWE tasks (SWE-bench Verified).
The work also assessed training-free and lightweight learned compression attempts, and documented efficiency effects and limits.

\section{Summary of Contributions}

\begin{enumerate}
	\item \textbf{An applied methodology for agentic context condensation.}
	This work details the implementation of a compressor--decoder pipeline based on the In-Context Autoencoder (ICAE) framework (Figure~\ref{fig:icae}), specifically adapted for managing the long observational contexts in agentic workflows (Figure~\ref{fig:icae-agent-training-overview}).
	This provides a reproducible template for researchers investigating implicit compression as an alternative to retrieval-based or extended-context models.
	
	\item \textbf{A multi-domain evaluation of implicit compression.}
	The method was benchmarked across three distinct domains: open-domain question answering (SQuAD, see Table~\ref{tab:icae_squad}), code reconstruction (RepoQA, see Figure~\ref{fig:repoqa_metrics}), and agentic software engineering (SWE-bench Verified, see Table~\ref{tab:qwen_icae_variants_absolute}).
	This comprehensive evaluation offers a granular understanding of the trade-offs involved, suggesting that performance on standard NLP tasks does not necessarily transfer to multi-turn, interactive agentic settings.
	
	\item \textbf{Quantification of performance trade-offs.}
	The empirical results establish a trade-off.
	ICAE improves question-answering accuracy (Table~\ref{tab:icae_squad}) and allows for longer agent trajectories (Figure~\ref{fig:boxplot-stepcount}) with a modest generational time reduction (Table~\ref{tab:qwen_icae_variants_absolute}).
	However, under the tested configuration, it degrades end-to-end task resolution on SWE-bench Verified (Table~\ref{tab:qwen_icae_variants_absolute}).
	This suggests that for other researchers, the value of this compression approach is task-dependent, highlighting a critical challenge: maintaining high-fidelity information in compressed state representations for multi-step tasks.
	
	\item \textbf{Exclusion of simpler compression alternatives.}
	The research also documents the failure of simpler, training-free condensation methods (e.g., averaging, as shown in Table~\ref{tab:avg_variants}) and lightweight projectors (Figure~\ref{fig:squad_metrics}).
	This finding steers future work away from trivial baselines and reinforces the need for expressive compressors to create useful context representations.
	
	\item \textbf{Publicly available implementation and experimental artifacts.}
	A modular reimplementation of the framework, along with training configurations and logs (see Appendix~\ref{app:training_details}), has been made available.
	This facilitates direct replication of the presented results and provides a foundation for extensions, such as exploring different backbone models, training procedures, or more advanced compression architectures.
\end{enumerate}



\section{Synthesis of Findings}

\textbf{RQ1 --- Efficiency: Longer trajectories and generation speed.}
Implicit context condensation increased the number of agent steps before context overflow (113 vs.\ 81 on average, a 40\% increase) and reduced per-call generation time by approximately 10\% (1.12 seconds vs.\ 1.23 seconds for the baseline).
The compressed agent's total time comprises 0.31 seconds for observation compression and 0.81 seconds for tool call generation.
Thus, the method enables the completion of longer trajectories under a fixed context window, thereby improving efficiency.

\textbf{RQ2 --- Transferability: NLP to Agentic SWE.}
Performance gains observed on SQuAD and preserved fidelity on RepoQA indicate successful transfer to single-shot software engineering tasks.
However, these benefits did not extend to the agentic setup in end-to-end SWE-bench Verified (\Cref{tab:qwen_icae_variants_absolute}; \Cref{fig:boxplot-stepcount}).
The model learned to predict next actions well on static trajectories (higher BLEU) yet under-resolved tasks when its own actions shaped subsequent observations (\Cref{fig:repoqa_metrics} vs.\ \Cref{fig:bleu-boxplot-combined}).
This gap highlights the difference between offline imitation or single-shot tasks and online, multi-turn control in software engineering environments (\Cref{sec:rq2_performance_domains}).

\textbf{Overall interpretation.}
ICAE-style implicit condensation is suitable when the goal is to read more context or run longer with moderate accuracy demands, and it is effective for extractive QA and single-shot SWE tasks with task-specific fine-tuning.
In contrast, for agentic SWE tasks where agents must plan, act, and recover from their own intermediate outputs, the present configuration ($4\times$ compression; LoRA-only encoder; disabled reasoning) shows a decrease in the ability to solve the tasks end-to-end.
The limitations chapter lists concrete avenues for future research (full-parameter fine-tuning of the encoder, higher-fidelity AE for code, adaptive memory sizing, enabling reasoning) that follow directly from the observed failure modes and should likely address the current shortcomings (\Cref{sec:limitations}, \Cref{sec:kv_caching}, \Cref{sec:comp_resources}).

In summary, the approach enables agents to run longer and faster, is applicable for QA and single-shot coding tasks, but does not improve (and, in this setup, actually harms) agentic end-to-end SWE-bench resolve rates.
% ========================================
% APPENDIX SECTION
% ========================================
\appendix
\chapter{Appendix}


% ========================================
% APPENDIX SECTION A.1: TRAINING DETAILS AND HYPERPARAMETERS
% ========================================
\section{Training Details and Hyperparameters}

Detailed configuration for training (Optimizer, learning rate, batch size, number of steps for pretrain and fine-tuning). Information on hardware used (Nvidia H200 GPUs).


% ========================================
% APPENDIX SECTION A.2: PROFILING SETUP AND LATENCY MEASUREMENT
% ========================================
\section{Profiling Setup and Latency Measurement}

Technical details of the test machine and runtime configuration used for latency measurements.


% ========================================
% APPENDIX SECTION A.3: DATASET CONSTRUCTION DETAILS (PWC)
% ========================================
\section{Dataset Construction Details (PWC)}

Details on the creation of the PWC dataset using GPT-4 to generate (context, prompt, answer) triples, including the prompt used for generation.


% ========================================
% APPENDIX SECTION A.4: DETAILED EVALUATION TABLES
% ========================================
\section{Detailed Evaluation Tables}

Comprehensive tables of model performance, including token-wise accuracy, mean tool-call time, and resolved issues for various ICAE variants (e.g., Qwen-LoRA-FT, ICAE (Qwen-LoRA-FT) Qwen).


% ========================================
% APPENDIX SECTION A.5: CODE AND REPRODUCIBILITY
% ========================================
\section{Code and Reproducibility}

Note on the necessity of hosting all code on GitHub to ensure reproducibility.


% ========================================
% BACKMATTER SECTION
% ========================================
% Lists of figures and tables (may be removed if not needed)
\backmatter{}
\listoffigures% may be removed
\listoftables% may be removed

% ========================================
% BIBLIOGRAPHY SECTION
% ========================================
% Add any additional citations that haven't been referenced in the text
\nocite{*} % include all bibliography entries
\printbibliography{} % print bibliography

\end{document}

%%% Local Variables:
%%% mode: latex
%%% TeX-engine: default
%%% TeX-command-extra-options: "-shell-escape"
%%% ispell-local-dictionary: "american"
%%% eval: (setenv "TEXINPUTS" ".//:")
%%% TeX-master: t
%%% End:
