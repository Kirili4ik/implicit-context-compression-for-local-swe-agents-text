% ========================================
% APPENDIX SECTION
% ========================================
\appendix
\chapter{Appendix}


% ========================================
% APPENDIX SECTION A.1: TRAINING DETAILS AND HYPERPARAMETERS
% ========================================
\section{On the Use of AI}

I \textbf{have} used generative AI to help me write the text for this work.\\
I \textbf{have} used generative AI to help me write the code for the experiments.\\
I \textbf{have not} used AI in order to create new experiments, nor for any goals, research questions, hypotheses, etc.


% ========================================
% APPENDIX SECTION A.0: SWE-SMITH PROMPT
% ========================================
\section{SWE-smith Prompt}
\label{app:swe-smith-prompt}

\begin{tcolorbox}[title={SWE-smith Prompt},
    colback=white, colframe=black, fonttitle=\bfseries,
    breakable, sharp corners=south, enhanced jigsaw]
  \begin{Verbatim}[breaklines=true, obeytabs=false,breaksymbol={},breakindent=0pt,fontsize=\tiny]
    You are a helpful assistant that can interact with a computer to solve tasks.
    <IMPORTANT>
    * If user provides a path, you should NOT assume it's relative to the current working directory. Instead, you should explore the file system to find the file before working on it.
    </IMPORTANT>
    
    You have access to the following functions:
    
    ---- BEGIN FUNCTION #1: bash ----
    Description: Execute a bash command in the terminal.
    
    Parameters:
      (1) command (string, required): The bash command to execute. Can be empty to view additional logs when previous exit code is `-1`. Can be `ctrl+c` to interrupt the currently running process.
    ---- END FUNCTION #1 ----
    
    ---- BEGIN FUNCTION #2: submit ----
    Description: Finish the interaction when the task is complete OR if the assistant cannot proceed further with the task.
    No parameters are required for this function.
    ---- END FUNCTION #2 ----
    
    ---- BEGIN FUNCTION #3: str_replace_editor ----
    Description: Custom editing tool for viewing, creating and editing files
    * State is persistent across command calls and discussions with the user
    * If `path` is a file, `view` displays the result of applying `cat -n`. If `path` is a directory, `view` lists non-hidden files and directories up to 2 levels deep
    * The `create` command cannot be used if the specified `path` already exists as a file
    * If a `command` generates a long output, it will be truncated and marked with `<response clipped>`
    * The `undo_edit` command will revert the last edit made to the file at `path`
    
    Notes for using the `str_replace` command:
    * The `old_str` parameter should match EXACTLY one or more consecutive lines from the original file. Be mindful of whitespaces!
    * If the `old_str` parameter is not unique in the file, the replacement will not be performed. Make sure to include enough context in `old_str` to make it unique
    * The `new_str` parameter should contain the edited lines that should replace the `old_str`
    
    Parameters:
      (1) command (string, required): The commands to run. Allowed options are: `view`, `create`, `str_replace`, `insert`, `undo_edit`.
    Allowed values: [`view`, `create`, `str_replace`, `insert`, `undo_edit`]
      (2) path (string, required): Absolute path to file or directory, e.g. `/repo/file.py` or `/repo`.
      (3) file_text (string, optional): Required parameter of `create` command, with the content of the file to be created.
      (4) old_str (string, optional): Required parameter of `str_replace` command containing the string in `path` to replace.
      (5) new_str (string, optional): Optional parameter of `str_replace` command containing the new string (if not given, no string will be added). Required parameter of `insert` command containing the string to insert.
      (6) insert_line (integer, optional): Required parameter of `insert` command. The `new_str` will be inserted AFTER the line `insert_line` of `path`.
      (7) view_range (array, optional): Optional parameter of `view` command when `path` points to a file. If none is given, the full file is shown. If provided, the file will be shown in the indicated line number range, e.g. [11, 12] will show lines 11 and 12. Indexing at 1 to start. Setting [start_line, -1] shows all lines from start_line to the end of the file.
    ---- END FUNCTION #3 ----
    
    
    If you choose to call a function ONLY reply in the following format with NO suffix:
    
    Provide any reasoning for the function call here.
    <function=example_function_name>
    <parameter=example_parameter_1>value_1</parameter>
    <parameter=example_parameter_2>
    This is the value for the second parameter
    that can span
    multiple lines
    </parameter>
    </function>
    
    <IMPORTANT>
    Reminder:
    - Function calls MUST follow the specified format, start with <function= and end with </function>
    - Required parameters MUST be specified
    - Only call one function at a time
    - Always provide reasoning for your function call in natural language BEFORE the function call (not after)
    </IMPORTANT>
  \end{Verbatim}
  \end{tcolorbox}

\section{Training Details and Hyperparameters}
\label{app:training_details}

\subsection{Pretraining Configuration}

\begin{table}[h]
    \centering
    \small
    \begin{tabular}{ll}
        \toprule
        \textbf{Parameter} & \textbf{Value} \\
        \midrule
        Base Model & Qwen3-8B \\
        Dataset & SlimPajama-6B \\
        Learning Rate & $1 \times 10^{-4}$ \\
        Batch Size & 1 \\
        Gradient Accumulation & 8 \\
        Training Steps & $\approx$100,000 \\
        Memory Size & 256 tokens (4× compression) \\
        LoRA Rank & 128 \\
        LoRA Target Modules & q\_proj, v\_proj \\
        Optimizer & AdamW \\
        Warmup Steps & 300 \\
        Hardware & 1× NVIDIA H200 GPU \\
        Training Time & $\approx$1 day 15 hours \\
        \bottomrule
    \end{tabular}
    \caption{Pretraining hyperparameters and configuration}
    \label{tab:pretrain_config}
\end{table}

\subsection{Fine-tuning Configuration}

\begin{table}[h]
    \centering
    \small
    \begin{tabular}{ll}
        \toprule
        \textbf{Parameter} & \textbf{Value} \\
        \midrule
        Base Model & Qwen3-8B \\
        Dataset & SWE-bench trajectories \\
        Learning Rate & $5 \times 10^{-5}$ \\
        Batch Size & 1 \\
        Gradient Accumulation & 1 \\
        Training Steps & $\approx$150,000 \\
        Memory Size & 256 tokens (4× compression) \\
        LoRA Rank & 128 \\
        LoRA Target Modules & q\_proj, v\_proj \\
        Optimizer & AdamW \\
        Warmup Steps & 250 \\
        Hardware & 1× NVIDIA H200 GPU \\
        Training Time & $\approx$3 days \\
        \bottomrule
    \end{tabular}
    \caption{Fine-tuning hyperparameters and configuration}
    \label{tab:finetune_config}
\end{table}

\subsection{Reproducibility Resources}

To ensure full reproducibility, we publish our complete implementation including:
\begin{itemize}
    \item Complete ICAE framework for both pretraining and fine-tuning phases (\url{https://github.com/JetBrains-Research/icae})
    \item Full Weights \& Biases experiment logs for pretraining: \url{https://wandb.ai/kirili4ik/icae-pretraining}
    \item Full Weights \& Biases experiment logs for fine-tuning: \url{https://wandb.ai/kirili4ik/icae-swebench-finetune}
    \item Pretrained model checkpoints achieving 95\% reconstruction BLEU: TODO
    \item Fine-tuned models that outperform uncompressed baselines on SQuAD: TODO
\end{itemize}


% ========================================
% APPENDIX SECTION A.2: PROFILING SETUP AND LATENCY MEASUREMENT
% ========================================
\section{Profiling Setup and Latency Measurement}

Technical details of the test machine and runtime configuration used for latency measurements.


% ========================================
% APPENDIX SECTION A.4: DETAILED EVALUATION TABLES
% ========================================
\section{Detailed Evaluation Tables}

Comprehensive tables of model performance, including token-wise accuracy, mean tool-call time, and resolved issues for various ICAE \cite{ge_-context_2024} variants (e.g., Qwen-LoRA-FT, ICAE (Qwen-LoRA-FT) Qwen).
