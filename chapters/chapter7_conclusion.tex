% ========================================
% CHAPTER 7: CONCLUSION AND OUTLOOK (VERY HIGH LEVEL ALL)
% ========================================
\chapter{Conclusion and Outlook}


% ========================================
% SECTION 7.1: SUMMARY OF ACHIEVEMENTS
% ========================================
\section{Summary of Achievements}

The ICAE \cite{ge_-context_2024} framework successfully achieves context condensation/feature extraction for general text, demonstrating high reconstruction quality (BLEU $\approx 99\%$) and measurable efficiency gains (speedup 2× to 3.6×).
The work provided insights into LLM memorization patterns, suggesting similarities to human memory encoding.


% ========================================
% SECTION 7.2: SYNTHESIS OF FINDINGS
% ========================================
\section{Synthesis of Findings}

Despite achieving efficiency and local accuracy on agent trajectories, the performance degradation in end-to-end task completion (resolved issues) highlights the critical gap between local context compression quality and robust decision-making in complex agentic settings.


% ========================================
% SECTION 7.3: POSITIONING THE WORK
% ========================================
\section{Positioning the Work}

The findings position this work within the broader research efforts on LLM context management, emphasizing the experimental results concerning condensation effectiveness across different data types, and suggesting caution when applying general compression methods to fine-grained, critical agent behaviors (code and tool use).
