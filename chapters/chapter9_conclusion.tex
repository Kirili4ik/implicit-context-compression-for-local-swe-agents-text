\chapter{Conclusion}
\label{cha:conclusion}

This thesis examined implicit context condensation for software engineering agents, instantiating the approach with an In-Context Autoencoder (ICAE) built on Qwen3-8B and evaluating it on question answering (SQuAD), code reconstruction (RepoQA), and end-to-end agentic SWE tasks (SWE-bench Verified).
The work also assessed training-free and lightweight learned compression attempts, and documented efficiency effects and limits.

\section{Summary of Contributions}

\begin{enumerate}
	\item \textbf{An applied methodology for agentic context condensation.}
	This work details the implementation of a compressor--decoder pipeline based on the In-Context Autoencoder (ICAE) framework (Figure~\ref{fig:icae}), specifically adapted for managing the long observational contexts in agentic workflows (Figure~\ref{fig:icae-agent-training-overview}).
	This provides a reproducible template for researchers investigating implicit compression as an alternative to retrieval-based or extended-context models.
	
	\item \textbf{A multi-domain evaluation of implicit compression.}
	The method was benchmarked across three distinct domains: open-domain question answering (SQuAD, see Table~\ref{tab:icae_squad}), code reconstruction (RepoQA, see Figure~\ref{fig:repoqa_metrics}), and agentic software engineering (SWE-bench Verified, see Table~\ref{tab:qwen_icae_variants_absolute}).
	This comprehensive evaluation offers a granular understanding of the trade-offs involved, suggesting that performance on standard NLP tasks does not necessarily transfer to multi-turn, interactive agentic settings.
	
	\item \textbf{Quantification of performance trade-offs.}
	The empirical results establish a trade-off.
	ICAE improves question-answering accuracy (Table~\ref{tab:icae_squad}) and allows for longer agent trajectories (Figure~\ref{fig:boxplot-stepcount}) with a modest generational time reduction (Table~\ref{tab:qwen_icae_variants_absolute}).
	However, under the tested configuration, it degrades end-to-end task resolution on SWE-bench Verified (Table~\ref{tab:qwen_icae_variants_absolute}).
	This suggests that for other researchers, the value of this compression approach is task-dependent, highlighting a critical challenge: maintaining high-fidelity information in compressed state representations for multi-step tasks.
	
	\item \textbf{Exclusion of simpler compression alternatives.}
	The research also documents the failure of simpler, training-free condensation methods (e.g., averaging, as shown in Table~\ref{tab:avg_variants}) and lightweight projectors (Figure~\ref{fig:squad_metrics}).
	This finding steers future work away from trivial baselines and reinforces the need for expressive compressors to create useful context representations.
	
	\item \textbf{Publicly available implementation and experimental artifacts.}
	A modular reimplementation of the framework, along with training configurations and logs (see Appendix~\ref{app:training_details}), has been made available.
	This facilitates direct replication of the presented results and provides a foundation for extensions, such as exploring different backbone models, training procedures, or more advanced compression architectures.
\end{enumerate}



\section{Synthesis of Findings}

\textbf{RQ1 --- Efficiency: Longer trajectories and generation speed.}
Implicit context condensation increased the number of agent steps before context overflow (113 vs.\ 81 on average, a 40\% increase) and reduced per-call generation time by approximately 10\% (1.12 seconds vs.\ 1.23 seconds for the baseline).
The compressed agent's total time comprises 0.31 seconds for observation compression and 0.81 seconds for tool call generation.
Thus, the method enables the completion of longer trajectories under a fixed context window, thereby improving efficiency.

\textbf{RQ2 --- Transferability: NLP to Agentic SWE.}
Performance gains observed on SQuAD and preserved fidelity on RepoQA indicate successful transfer to single-shot software engineering tasks.
However, these benefits did not extend to the agentic setup in end-to-end SWE-bench Verified (\Cref{tab:qwen_icae_variants_absolute}; \Cref{fig:boxplot-stepcount}).
The model learned to predict next actions well on static trajectories (higher BLEU) yet under-resolved tasks when its own actions shaped subsequent observations (\Cref{fig:repoqa_metrics} vs.\ \Cref{fig:bleu-boxplot-combined}).
This gap highlights the difference between offline imitation or single-shot tasks and online, multi-turn control in software engineering environments (\Cref{sec:rq2_performance_domains}).

\textbf{Overall interpretation.}
ICAE-style implicit condensation is suitable when the goal is to read more context or run longer with moderate accuracy demands, and it is effective for extractive QA and single-shot SWE tasks with task-specific fine-tuning.
In contrast, for agentic SWE tasks where agents must plan, act, and recover from their own intermediate outputs, the present configuration ($4\times$ compression; LoRA-only encoder; disabled reasoning) shows a decrease in the ability to solve the tasks end-to-end.
The limitations chapter lists concrete avenues for future research (full-parameter fine-tuning of the encoder, higher-fidelity AE for code, adaptive memory sizing, enabling reasoning) that follow directly from the observed failure modes and should likely address the current shortcomings (\Cref{sec:limitations}, \Cref{sec:kv_caching}, \Cref{sec:comp_resources}).

In summary, the approach enables agents to run longer and faster, is applicable for QA and single-shot coding tasks, but does not improve (and, in this setup, actually harms) agentic end-to-end SWE-bench resolve rates.