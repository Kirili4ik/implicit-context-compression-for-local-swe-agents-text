\chapter{Evaluation}
\label{cha:evaluation}

\section{Datasets}
\label{sec:datasets}

\subsection{SWE-bench}
We have chosen to use the SWE-bench \cite{jimenez2024swebench} dataset for our experiments.
It is a well known dataset for evaluating the performance of SWE agents.
It contains a large number of SWE tasks, each with a set of instructions and a set of expected outputs.
They were collected from real life GitHub issues. 
We have chosen to work with a subset -- SWE-bench Verified \cite{swebench-verified}.
It is a subset of the SWE-bench dataset that contains only the tasks that have been verified to be correct.

\subsection{Agentic Trajectories as Data}
We treat sequential action--observation interactions (trajectories) as training data.
These trajectories were obtained using a strong teacher model (e.g., Claude Sonnet 3.7) on the SWE-bench Verified dataset to produce high-quality inputs suitable for futher training.
The setup for generating these trajectories follows the SWE-smith setup closely\cite{swe-smith}.

\subsection{SQuAD}
The Stanford Question Answering Dataset (SQuAD) is a widely used reading comprehension dataset, consisting of questions posed by crowdworkers on a set of Wikipedia articles, where the answer to every question is a segment of text, or span, from the corresponding reading passage.
wriote here shortly about https://arxiv.org/pdf/1806.03822

\subsection{RepoQA \cite{liu2024repoqa}}
RepoQA \cite{liu2024repoqa} is a benchmark designed to evaluate a model's ability to understand and reason about code at the repository level.
Unlike traditional code-related tasks that focus on standalone snippets, RepoQA requires a holistic understanding of entire codebases to answer questions.
The dataset consists of question-answer pairs grounded in real-world open-source repositories, covering a wide range of topics from API usage to intricate implementation details.
This makes it an ideal benchmark for testing the limits of our ICAE model, as it assesses the ability to compress and retrieve high-fidelity information from long, structured, and semantically dense code contexts.

\section{Quality Metrics}
\label{sec:quality_metrics}

\subsection{Some Quality Metrics}
We report several metrics to evaluate our approach.
As a simple proxy metric, we measure token-wise accuracy —- averaged fraction of the guessed tokens that match the reference trajectory (note: this is measured with teacher forcing).
However, the most important metric is the number of successfully resolved issues on SWE-bench Verified, which directly reflects the model's ability to complete real-world software engineering tasks.
We also measure mean tool-call generation time to assess computational performance.

We note that measuring trajectory length is not particularly meaningful in our setting, as 8B-scale models frequently enter loops where they repeatedly call the same tool, artificially inflating trajectory length without making meaningful progress toward task completion.

write here about autoregressive accuracy and tha it is in a way better than token-wise teacher forced accuracy (in teacher forcing its about 90 and in autoregressive its about 10 so its like closer to reality)

\subsection{BLEU Score}
The Bilingual Evaluation Understudy (BLEU) score is a metric for evaluating the quality of text which has been machine-translated from one natural language to another.
BLEU's output is always a number between 0 and 1. This value indicates how similar the candidate text is to the reference texts, with values closer to 1 representing more similar texts.
It was one of the first metrics to claim a high correlation with human judgements of quality, and remains one of the most popular automated and inexpensive metrics.

In our context, we use the BLEU score to evaluate the quality of the reconstructed text after compression and decompression by the ICAE model.
A high BLEU score indicates that the reconstructed text is very similar to the original text, which means that the compression is nearly lossless.
This is particularly important for code, where even small changes can alter the semantics of the program.
